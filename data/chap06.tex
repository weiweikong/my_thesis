\chapter{基于多传感器的无人机引导系统设计}

\section{基于激光反射器的引导系统设计}

\section{基于超宽带雷达的引导系统设计}

\section{基于EKF的转台信息滤波问题}
位姿估计问题(Pose Estimation)是估计特定物体相对于参考坐标系的位置和姿态。在传统方法中,GPS、惯性测量单元(IMU)、视觉、激光雷达等作为主要传感器来对目标进行估计和判断。位子估计问题既可以只依赖视觉传感器,也可以综合应用各类型传感器采集到的数据通过滤波得到。在以视觉信息为主的位姿估计方法中,主要有一下两类方法:(1)单目视觉。单视觉方法主要依赖对平面的检测、合作标志的检测、海天线或地平面的检测。(2)多目视觉。多目视觉方法主要指针对特征点在不同成像平面的位置,通过多个相机的排布,从而解算出目标的位置和姿态信息。

基于视觉的位姿估计问题主要应用于人机交互(Human Computer Interaction)、视觉里程计(VO)和器人导航和实时定位与建图(SLAM)。其中,视觉里程计主要通过对系列连续图像进行处理,结合飞行器自身携带的各类传感器,最终得到可靠的三维信息(3D motion)。针对长基线系统(如图\ref{fig:38_LangBaseLineSystem}所示),利用EKF方法,将转台信息和图像信息相融合,可以得到较好的目标位置状态。定义$X$是状态模型
\begin{equation}
X=[Pos^s, V^s, C^{Att}, w^{Att}]^T
\end{equation}
其中
$$
Pos^s=\left[\begin{array}{c}
x_l\\
y_l\\
z_l\\
\end{array}\right],
V^s=\left[\begin{array}{c}
v_x\\
v_y\\
v_z\\
\end{array}\right]
$$
$$
C^{Att}=\left[\begin{array}{c}
lpan\\
ltilt\\
rpan\\
rtilt\\
\end{array}\right],
w^{Att}=\left[\begin{array}{c}
wlpan\\
wltilt\\
wrpan\\
wrtilt\\
\end{array}\right].
$$
上述所有变量均在世界坐标系下,其中$lpan$ 和 $ltilt$ 表示左侧相机的俯仰角度和水平角度; $rpan$ 和 $rtilt$ 对应右侧相机的两个角度; $wlpan, wltilt, wrpan, wrtilt$ 表示上述四个角度的角速度。
在第$k$次迭代过程中,状态转换方程如下
\begin{equation}
\bar{x}_{k|k_1}=F_k\bar{x}_{k-1|k-1}
\end{equation}
其中$F_k$是状态转换矩阵。根据EKF的基本假设,在第$k$时刻的状态只从第$k-1$时刻得到。因此可以得到目标的预测位置和相机的姿态解算公式,
\begin{equation}
\overline{Pos}^s_{k|k-1}=\Delta t V^s_{k-1|k-1} + {Pos}^s_{k-1|k-1}
\end{equation}
\begin{equation}
\overline{C}^{Att}_{k|k-1}=\Delta t w^{Att}_{k-1|k-1} + \overline{C}^{Att}s_{k-1|k-1}
\end{equation}
其中$\Delta t$表示相邻两次时间间隔。状态转换方程可以写为,
\begin{equation}
\bar{x}_{k|k-1}=\left[\begin{matrix}
I_{3\times 3} & \Delta t_{3\times 3} & \textbf{0}_{3\times 4} & \textbf{0}_{3\times 4}\\
\textbf{0}_{3\times 3} & I_{3\times 3} & \textbf{0}_{3\times 4} & \textbf{0}_{3\times 4}\\
\textbf{0}_{4\times 3} & \textbf{0}_{4\times 3} & I_{4\times 4} & \Delta t_{4\times 4}\\
\textbf{0}_{4\times 3} & \textbf{0}_{4\times 3} & \textbf{0}_{4\times 4} & I_{4\times 4}\\
\end{matrix}\right]\bar{x}_{k-1|k-1}
\end{equation}
其中$\Delta t_{n \times n}$表示$n \times n$阶矩阵,其对角线位置为$\Delta t$,其余位置为0。在第$k$时刻,系统的协方差矩阵为,
\begin{equation}
P_{k|k-1}=F_kP_{k-1|k-1}F^T_k+G_kQ_kG^T_k
\end{equation}
其中$Q_k$是动态系统的高斯噪声,$G_k$是第$k$时刻$Q_k$的Jacobian矩阵。状态协方差矩阵由四个部分组成,分别对应:目标-目标、目标-相机、相机-相机和相机-目标,其公式为
\begin{equation}
P_k=\left[
\begin{matrix}
P_{T|T} & P_{T|C} \\
P_{C|T} & P_{C|C} \\
\end{matrix}\right].
\end{equation}

EKF的观测模型基于传统的相机成像模型和映射模型,测量向量$z$为,
\begin{equation}
z=[Pos^I_L, Pos^I_R, C^{PTU}]^T
\end{equation} 
其中$Pos^I_L$和$Pos^I_R$分别表示目标在左右相机的位置,$(u^L_k, v^L_k)$和$(u^R_k, v^R_k)$表示目标在相平面的位置。考虑到相机模型为典型的针孔模型,目标位置$\overline{Pos}^s_{k|k-1}$通过转换从世界坐标系到图像坐标系。这里,假设相机光心位置和PTU转台位置在运动过程中保持不变,即忽略安装偏移误差。为了计算目标在相平面的位置,将相机位置、内参数等引入方程,
\begin{equation}
\left[\begin{matrix}
\overline{Pos}^I \\
1\\
\end{matrix}\right]
=h(\overline{Pos}^s_{k|k-1}, \overline{C}^{Att}_{k|k-1})
=\lambda N^{in}\overline{M}^{out}_{k|k-1}\overline{Pos}^s_{k|k-1}
\end{equation}
其中$\lambda$是归一化参数。$N^{in}$和$\overline{M}^{out}_{k|k-1}$是转换矩阵。其他参数可以通过如下方程表达,
\begin{equation}
\overline{M}^{out}_{k|k-1}
= \left[
\begin{matrix}
\bar{R}^c_{k|k-1} & T \\
\textbf{0}^T_3 & 1 \\
\end{matrix}
\right]
\end{equation}
和
\begin{equation}
N^{in}=
\left[
\begin{matrix}
1/d_x & 0 & u_o \\
0 & 1/d_y & v_0 \\
0 & 0 & 1 \\
\end{matrix}
\right]
\left[
\begin{matrix}
f & 0 & 0 & 0 \\
0 & f & 0 & 0 \\
0 & 0 & 0 & 1 \\
\end{matrix}
\right]
\end{equation}
其中$\bar{R}^c_{k|k-1}$是旋转矩阵$\overline{C}^{Att}_{k|k-1}$和$T$表达相机坐标系和世界坐标系之间的位置关系。至此,可以得到观测模型,
\begin{equation}
\bar{z_k}=
\left[
\begin{matrix}
\overline{Pos}^I_L \\
\overline{Pos}^I_R \\
\overline{C}^{PTU} \\
\end{matrix}
\right]
=
\left[
\begin{matrix}
Block (\lambda_L N^{in}_L \overline{M}^{out}_{L(k|k-1)}\overline{Pos}^s_{k|k-1} )\\
Block (\lambda_R N^{in}_R\overline{M}^{out}_{R(k|k-1)}\overline{Pos}^s_{k|k-1}  )\\
I_{4 \times 4}\overline{C}^{Att}_{k|k-1} \\
\end{matrix}
\right]
\end{equation}
定义$Block(\cdot)$作为仅提取矩阵前两项的函数。


$x_k$是$z_k$的更新结果。根据观测模型$h$,其Jacobian矩阵$H_k$可以表达为
\begin{equation}
H_k=\frac{\partial h ( \overline{Pos}^S_{k|k-1}, \overline{C}^{Att}_{k|k-1} )}{\partial x_k} 
\end{equation}
得到卡尔曼增益$K_t$,
\begin{equation}
S_k=H_k P_{k|k-1} H^T_k + R
\end{equation}
\begin{equation}
K_k=P_{k|k-1}H^T_k(S_k)^{-1}
\end{equation}
其中$R$是传感器的高斯白噪声。最后,得到更新方程,
\begin{equation}
\bar{x}_{k|k} = \bar{x}_{k|k-1}+K_k(z_k-h(\bar{x}_{k|k-1}))
\end{equation}
\begin{equation}
P_{k|k}=(1-K_k H_k)P_{k|k-1}
\end{equation}
目标识别方法的输出做为EKF的输入,随后将EKF输出结果返回给无人机。目前,融合算法只是用了转台和可见光相机信息,后续还要引入飞机姿态信息和降落平台姿态信息,以便得到更准确的位置和姿态信息。

\section{舰船运动估计模型}
 
 
 
\section{无人机降落决策系统}
一般而言,无人机的最终决断时间在接触甲板之前的1.5s至12.5s。定义下滑曲线的方位角误差为$\delta_h$,距离跑道中心线的误差为$\epsilon_h$,下滑曲线的俯仰角误差为$\delta_v$,距离期望下降曲线的高度差为$\epsilon_v$。无人机降落决策系统主要是根据无人机下降阶段的六个基本数据做出是否复飞的判断。这六个量分别为
\begin{compactenum}
	\item 跑道中心线的误差为$\epsilon_h$
\item 距离期望下降曲线的高度差为$\epsilon_v$
\item 无人机横滚角$\phi$
\item 无人机俯仰角$\theta$
\item 航母甲板下沉速率$\dot{h_c}$
\item 航母甲板侧滑速率$v_s$
\end{compactenum}
根据上述六个不同的基本数据指标,无人机主要面临一些六种危险如下表所示。
\begin{table}
	\centering
	\begin{tabular}{cccc}
		\hline
		风险名称   & 符号表达 & 相关变量 & 状态描述       \\ \hline
		撞击甲板下方 & PRS  & $e\_v$ & 超过纵向最大负阈值  \\
		未接触拦阻索 & PUL  & $e\_v$ & 超过纵向最大正阈值  \\
		横向偏差大  & POL  & 无   & 超过横向最大正负阈值 \\
		姿态偏差大  & PBA  & 无    & 着舰姿态偏差大    \\
		硬着舰    & PHL  & 无   & 超过沉降最大阈值   \\
		横向偏离跑道 & PRB  & 无    & 无          \\
	\end{tabular}
\end{table}
其中$P_{RS}$定义为
\begin{equation}
P_{RS}=\frac{1}{\sigma_v\sqrt{2\pi}}\int_{-\infty}^{e_{v,low}}\exp(-\frac{(x-\bar{e}_v)^2}{2\sigma_v^2})dx \\
=\frac{1}{2}(1+\text{erf}(\frac{e_{v, low}-\bar{e}_v}{\sigma_v\sqrt{2}}))
\end{equation}
其中误差函数定义为
\begin{equation}
\text{erf}(x)=\frac{2}{\sqrt{\pi}}\int_{0}^{x}e^{-t^2}dt
\end{equation}
假设六种情况发生的概率是相互独立的,由此定义无人机正常回收、复飞和失败的概率。
\begin{align}
P_{recover}=(1-P_{RS})(1-P_{UL})(1-P_{OL})(1-P_{BA})(1-P_{HL})(1-P_{RB}) \\
P_{bolter}=(1-P_{RS})P_{UL}(1-P_{OL})(1-P_{BA})(1-P_{HL})(1-P_{RB}) \\
P_{fail}=1-(1-P_{RS})(1-P_{OL})(1-P_{BA})(1-P_{HL})(1-P_{RB})
\end{align}





 
 
%\begin{algorithm2e}
%	\SetAlgoLined
%	\SetKwInOut{Input}{Input}
%	\SetKwInOut{Output}{Output}
%	\Input{Initialized sequence $\hat{\psi}(k),\ \ k=1,2,..., N$}
% 	
%	\KwResult{$x(k)$}
%	%initialization\;
%	\While{$\Sigma |x_{error}| \le \alpha$ and $|x^{*(i)} - x^{*(i-1)} | \le \beta$}
%	{
%		Update UAV states \;
%		$p(k+1) \leftarrow$ Update($p(k)$)\;
%		$\dot{p}(k+1) \leftarrow$ Update($\dot{p}(k)$)\;
%		${\phi}(k+1) \leftarrow$ Update(${\phi}(k)$)\;
%		Solve SCP using CVXGEN Solver\;
%		\For{$j\leftarrow 1$ \KwTo $N-1$}{
%			Calculate $x_{error}$ \;
%		}
%		Update $\hat{\psi}(k)^{i+1} = \hat{\psi}(k)^{i} $\;
%		Update Trust Region, $\rho(k)^{i+1} = \gamma_{path}(k)^{i} \rho(k)^{i}$ \;
%		$i = i + 1$
%	}
%	\caption{侧风扰动下的最优轨迹求解}
%\end{algorithm2e}
%
%\begin{algorithm2e}[H]
%	\SetAlgoLined
%	\KwData{this text}
%	\KwResult{how to write algorithm with \LaTeX2e }
%	initialization\;
%	\While{not at end of this document}{
%		read current\;
%		\eIf{understand}{
%			go to next section\;
%			current section becomes this one\;
%		}{
%			go back to the beginning of current section\;
%		}
%
%	}
%	\caption{How to write algorithms}
%\end{algorithm2e}
%
%\begin{algorithm2e}
%	\SetKwData{Left}{left}\SetKwData{This}{this}\SetKwData{Up}{up}
%	\SetKwFunction{Union}{Union}\SetKwFunction{FindCompress}{FindCompress}
%	\SetKwInOut{Input}{Input}\SetKwInOut{Output}{Output}
%	\Input{A bitmap $Im$ of size $w\times l$}
%	\Output{A partition of the bitmap}
%	\BlankLine
%	\emph{special treatment of the first line}\;
%	\For{$i\leftarrow 2$ \KwTo $l$}{
%		\emph{special treatment of the first element of line $i$}\;
%		\For{$j\leftarrow 2$ \KwTo $w$}{\label{forins}
%			\Left$\leftarrow$ \FindCompress{$Im[i,j-1]$}\;
%			\Up$\leftarrow$ \FindCompress{$Im[i-1,]$}\;
%			\This$\leftarrow$ \FindCompress{$Im[i,j]$}\;
%			\If(\tcp*[h]{O(\Left,\This)==1}){\Left compatible with \This}{\label{lt}
%				\lIf{\Left $<$ \This}{\Union{\Left,\This}}
%				\lElse{\Union{\This,\Left}}
%			}
%			\If(\tcp*[f]{O(\Up,\This)==1}){\Up compatible with \This}{\label{ut}
%				\lIf{\Up $<$ \This}{\Union{\Up,\This}}
%				\tcp{\This is put under \Up to keep tree as flat as possible}\label{cmt}
%				\lElse{\Union{\This,\Up}}\tcp*[h]{\This linked to \Up}\label{lelse}
%			}
%		}
%		\lForEach{element $e$ of the line $i$}{\FindCompress{p}}
%	}
%	\caption{disjoint decomposition}\label{algo_disjdecomp}
% 	\caption{使用SCP方法求解侧风扰动下的最优轨迹}
% \end{algorithm2e}
% 

