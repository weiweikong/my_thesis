\chapter{无人机飞行制导与控制系统设计}
针对目标监测问题,很多学者提出了不同的解决方法,但目标实时监测问题中的所涉及的目标尺度变化、被遮挡和实时性等问题仍然没有得到有效解决。

\section{无人机降落轨迹设计方法}
%Ref: Automatic Ship Landing System for Fixed-Wing UAV

\subsection{基于直线的降落曲线规划}
基于连续离散点的路径规划最为直观,路径只需要将需要到达的点通过生成直线来连接,并在直线中产生相应的离散点即生成期望轨迹。定义任意一个航迹点所在直线表达为
\begin{align}
&x(\bar{\lambda}) =\bar{\lambda} a_x + b_x \\
&y(\bar{\lambda}) =  \bar{\lambda} a_y +b_y
\end{align}
其中$\bar{\lambda} \in [0, 1]$,在这段航迹的初始点$P_{start}(x_s, y_s)$时,$\bar{\lambda} = 0$;在这段航迹的结束点$P_{final}(x_f,y_f)$是,$\bar{\lambda} = 1$。由此,初始点和结束点的表达形式如下
\begin{align}
&P_{start} = [x(0)\ y(0)]^T=[b_x\ b_y]^T \\
&P_{final} = [x(1)\ y(1)]^T=[a_x+b_x\ a_y+b_y]^T
\end{align}
此时定义这段航迹相对于直线的正切角为
\begin{equation}
\psi_{path}=\text{atan2}(a_y, a_x)
\end{equation}
在依次给定每一段的航迹点$P_{start}$和$P_{final}$之后,通过上式即刻求得这段航迹的正切角度。在给定六个下降曲线的航迹点(距离期望着舰点的距离和高度)后,通过直线生成轨迹的方法,可以得到五段连续航线,如图\ref{fig:chp07_01_straight_line_plot_and_tangent}上半部分所示。这五段曲线的正切值的连续表达如图中下半部分所示。通过正切曲线可以看到,在航线之间的转换点,正切值出现明显突变。如果按此路径飞行,无人机的执行机构在航线转换点往往需要运动到极限位置,且容易引起系统的振荡和发散。因此该方法实际上无法用于无人机的路径规划和轨迹跟踪。注意,图\ref{fig:chp07_01_straight_line_plot_and_tangent}横坐标表达的是无人机的期望飞行轨迹,而不是该轨迹在惯性坐标系$\mathbf{i}^{i}$轴和$\mathbf{k}^{i}$轴所构成平面的投影长度。

\begin{figure}[ht]   
	\centering
	\includegraphics[width=\textwidth]{figs/chp07/chp07_01_straight_line_plot_and_tangent.pdf}
	\caption{基于直线的下降曲线规划示意图}
	\label{fig:chp07_01_straight_line_plot_and_tangent}
\end{figure}
由于基于直线连接的降落曲线的每一个航迹点之间不满足一阶微分可导的约束,因此纵向飞行控制器无法使得无人机无法按该期望轨迹飞行。曲线的连续性需要曲线的方程符合二阶处处可微的条件。但在实际飞行过程中,无人机飞行环境的复杂性以及无人机纵向控制器的控制量通常为下降速率,因此在设计下降曲线时可以弱化曲线的二阶处处可微的条件。


\subsection{基于Dubins曲线的下降阶段轨迹设计}
除直接在航迹点之间用直线连接生成轨迹之外,另一种常见轨迹生成方法是基于Dubins在1957年提出的Dubins Path\cite{dubins1957curves},该曲线最初主要是解决Dubins Car最短轨迹规划问题。该问题的前提假设是汽车只能向前运动。由于该假设与无人机的运动特性相符合,因此Dubins Path在无人机轨迹规划领域得到广泛应用。此外,Dubins也在文献\cite{dubins1957curves}中证明:在给定初始点位置和方向以及终止点位置和方向的情况下,Dubins Path生成的曲线是最短路径,且该路径规划由两个曲率最大的弧线和一条直线共三部分组成。

对于无人机下降过程中的轨迹而言,可以利用Dubin Path的设计方法,对上一节设计的航迹直线段进行修正,用曲线替换折线,从而得到理想的航迹切换曲线。一般而言,通过在两个航迹直接设定虚拟圆,使得无人机在到达目标点之前,根据当前的空速方向和下一个目标点与当前目标点连线的夹角,Dubins方法通常构造一个虚拟圆。无人机只需要在靠近下一个目标点之前,按虚拟圆的弧线进行飞行,即可实现平滑转弯。该虚拟圆能够保证与目标航点的前后两条直线航迹线相切。由于Dubins曲线设计方法考虑到运动学基本约束,一般为转弯曲率和运动速率,因此该方法生成的轨迹能够满足无人机降落曲线设计需求。

由于当前无人机位置和虚拟圆圆心位置的不同,Dubins的曲线生成一般有四种可能性,即RSR,RSL,LSR和LSL,具体内容可参考文献\cite{tsourdos2010cooperative}。本文选用RSR类型对问题进行描述,定义初始航点虚拟圆为$\mathbf{O}_{cs}(X_{cs},Y_{cs})$,结束航点虚拟圆为$\mathbf{O}_{cf}(X_{cf}, Y_{cf})$,两个虚拟圆圆心的求解公式如下
\begin{align}
&X_{cs} = X_s - R_s \cos (\psi_s \pm \frac{2}{\pi}) \\
&Y_{cs} = X_s - R_s \sin (\psi_f \pm \frac{2}{\pi}) \\
&X_{cf} = X_f - R_f \cos (\psi_f \pm \frac{2}{\pi}) \\
&Y_{cf} = X_f - R_f \sin (\psi_f \pm \frac{2}{\pi})
\end{align}
其中$R_s$和$R_f$是初始航点虚拟圆和结束航点虚拟圆的半径,$\psi_s$和$\psi_f$描述初始航点和虚拟航点在系统坐标系下的航向角。由此可以进一步得到两个圆心之间的夹角关系$\alpha_{path}$和圆心之间的连线与系统惯性坐标系之间的夹角$\beta_{path}$,
\begin{align}
&\alpha_{path} = \text{arcsin}( \frac{R_f-R_s}{|c|}) \\
&\beta_{path} = \text{arctan} (\frac{Y_{cf}-Y_{cs}}{X_{cf}-X_{cs}})
\end{align}
其中,$|c|= \parallel \mathbf{O}_{cs} - \mathbf{O}_{cf} \parallel $表示两个虚拟圆圆心之间的距离。图\ref{fig:chp07_02_dubins_RSR}所示为降落曲线下降阶段使用的$\text{RSR}$情况,即初始圆原点$\mathbf{O}_{cs}$位于初始速度$p_s$的右侧,结束圆点$\mathbf{O}_{cf}$位于结束期望速度$p_f$的右侧。
\begin{figure}[ht]   
	\centering
	\includegraphics[width=\textwidth]{figs/chp07/chp07_02_dubins_RSR.pdf}
	\caption{RSR类型的Dubins曲线示意图}
	\label{fig:chp07_02_dubins_RSR}
\end{figure}

考虑到无人机降落的危险性,并在降落过程中提高地面引导系统对无人机的识别,在固定点降落曲线设计过程中,本文主要设计六个基准点,下降曲线示意图如图\ref{fig:chp07_03_waypoints}所示。这六个航点和各个线段的基本定义与符号表达为:
\begin{compactenum}
	\item 第一航点(Way Point 1)位于回收网或期望着舰点的前方,该航点与第三航点(Way Point 3)之间的连接线穿过回收网的中间位置或者期望着舰点,该位置定义为第二航点(Way Point 2),这段航线与$\mathbf{i}^{path,l}$之间的夹角为$\gamma_1$。该角度定义为无人机触网或与期望着舰点所在平面的攻角,即无人机经过最后一航点时,无人机机体坐标系的$\mathbf{i}^{b}$轴并不与地面水平,无人机的攻角并不为零。一般定义第二航点的高度用$H_{td}$表达。
	\item 第三航点(Way Point 3)和第四航点(Way Point 4)之间设定一段平飞过程。一般而言,系统会在这个阶段对无人机状态和舰船状态进行最后检查,如果不满足系统的降落条件,则会将无人机的下降飞行模式切换为复飞模式。第三航点和第四航点之间的距离可以根据需求进行设定。
	\item 第四航点(Way Point 4)和第五航点(Way Point 5)构成下滑轨迹,其夹角定义为$\gamma_2$,描述无人机下降阶段下滑角度。一般可以将$\gamma_1$和$\gamma_2$中设定为相同的角度。
	\item 第六航点(Way Point 6)的高度与第五航点(Way Point 5)高度相同,距离进一步向远端延伸。这一航点为下降曲线的初始点,也是上一节进近曲线的结束点。该高度的设定还需要考虑引导系统的检测距离。检测距离的高度用$H_{detect}$表示。
	\item $a_1$、$a_2$、$a_3$、$a_4$和$a_5$分别为各航点在北向和南向所在平面的投影直接的距离。
\end{compactenum}

\begin{figure}[ht]   
	\centering
	\includegraphics[width=\textwidth]{figs/chp07/chp07_03_waypoints.pdf}
	\caption{六点下降轨迹设计示意图}
	\label{fig:chp07_03_waypoints}
\end{figure}



根据上述各航点的定义,首先定义在系统坐标系的航点表达方式
\begin{align}
\mathbf{WP}_n = (x_n^{path}\ y_n^{path} \ z_n^{path}\ \chi_n^{path})^T
\end{align}
其中$x_n^{path}, y_n^{path}, z_n^{path} \in \mathbb{R}^3$表示该航点的坐标,$\chi_n^{path}$表示无人机经过该航迹点的期望航迹航向角,$n$表示第$n$个航迹点,系统的第一个航迹点一般位于期望着舰位置之后,记为$\mathbf{WP}_1$,此后的航迹点从降落位置开始反向推算。由此,上述六个航点的数学表达为
\begin{align}
\mathbf{WP}_1 &= (-\frac{H_{td}}{\tan{\gamma_1}}\ 0 \ 0\ \chi_1^{path})^T \\
\mathbf{WP}_2 &= (0\ 0 \ -H_{td}\ \chi_2^{path})^T \\
\mathbf{WP}_3 &= \mathbf{WP}_2 + (\frac{H_{td}}{\tan{\gamma_1}}\ 0 \ {-H_{td}}\ \chi_3^{path})^T\\
\mathbf{WP}_4 &= \mathbf{WP}_3+ (a3 \ 0 \ 0\ \chi_4^{path})^T \\
\mathbf{WP}_5 &= \mathbf{WP}_4 + (\frac{H_{detect}}{\tan{\gamma_2}}\ 0 \ {-H_{detect}}\ \chi_5^{path})^T \\
\mathbf{WP}_6 &= \mathbf{WP}_5 + (a5\ 0 \ H_{detect}\ \chi_6^{path})^T 
\end{align}

由此,只需要给定期望着舰点高度$H_{td}$、着陆前末状态检查长度$a3$、降落位置攻角$\gamma_1$、下滑角$\gamma_2$和无人机检测高度$H_{detect}$,即可得到这六个航迹点在降落轨迹局部坐标系的坐标。假设期望着舰点$H_{td} = 3 m$,$\gamma_1=4 \degree$,$\gamma_2=8\degree$,$a_3 = 200\ m$和$H_{detect} = 50 m$,假设此时降落曲线局部坐标系的$\mathbf{i}^{path,l}$轴和$\mathbf{j}^{path,l}$轴方向分别于系统惯性系的$\mathbf{i}^{i}$和$\mathbf{j}^i$平行,六个航迹点的期望航迹航向角均为$-180\degree$,可以得到六个航点在系统坐标系的具体位置如表\ref{lab:uav_landing_six_points}所示。

\begin{table}[ht]
	\centering
	\caption{无人机下降曲线航点位置设定}
	\label{lab:uav_landing_six_points}
	\begin{tabular}{crrr}
		\hline
		& \multicolumn{1}{c}{X(m)} & \multicolumn{1}{c}{Y(m)} & \multicolumn{1}{c}{Z(m)} \\ \hline
		航点1 & 42.0 & 0.0 & 0.0 \\
		航点2 & 0.0 & 0.0 & -6.0 \\
		航点3 & 2.9 & 0.0 & -6.0 \\
		航点4 & 42.0 & 0.0 & -6.0 \\
		航点5 & 89.7 & 0.0 & -50.0 \\
		航点6 & 98.7 & 0.0 & -50.0 \\ \hline
	\end{tabular}
\end{table}
根据Dubins方法,可以将第三、第四和第五航迹点位置通过曲线进行平滑,得到期望的降落曲线下降阶段的期望轨迹,如图\ref{fig:chp07_04_landing_path_longitudinal_dubin}所示。
\begin{figure}[ht]   
	\centering
	\includegraphics[width=\textwidth]{figs/chp07/chp07_04_landing_path_longitudinal_dubin.pdf}
	\caption{基于Dubins方法设计的下降轨迹示意图}
	\label{fig:chp07_04_landing_path_longitudinal_dubin}
\end{figure}
其中蓝色点代表设计的六个不同位置的下降曲线航点;蓝色直线是航点之间的直线下降轨迹;紫色曲线是通过Dubins方法设计之后,平滑之后的下降轨迹;紫色三角和紫色方形标记表示Dubins方法生成的虚拟圆部分弧线段的起始位置和终止位置;红色五角星表示系统的期望着舰位置。


此外,通过使用Dubins方法进行轨迹设计,将曲线段引入到原始航点的初始位置和终止位置,使得轨迹处处一阶可导,无人机的下降速率不会出现突变,容易实现对无人机的稳定控制。这种方法的另一个优点是只需要设置最大曲率边界一个变量,即可生成相应的曲线。上述六个航点在不同位置的下降速率如图\ref{fig:chp07_05_landing_path_longitudinal_dubin_derivative}所示。
\begin{figure}[ht]   
	\centering
	\includegraphics[width=\textwidth]{figs/chp07/chp07_05_landing_path_longitudinal_dubin_derivative.pdf}
	\caption{下降轨迹速率}
	\label{fig:chp07_05_landing_path_longitudinal_dubin_derivative}
\end{figure}
由上图可以看到,通过设计下降曲线,可以估计不同位置无人机的下降速率,如果下降速率超过无人机的性能范围,可以将下滑角度进一步调整,以满足不同型号无人机的需求。

\subsection{基于Dubins曲线的进近阶段轨迹设计}
在进入上一节设计的下降轨迹之前,无人机需要从远端飞行到引导系统检测窗口。根据上一节的定义,该检测窗口的航迹点坐标就是第六个航迹点$\mathbf{WP}_6$,本节使用$P_f$符号进行标记。定义当前无人机在系统惯性坐标系下的位置为$P_s$。

根据Dubins方法,一般适用于二维平面的航迹点规划,由于无人机当前位置与检测窗口的高度存在一定的偏差,所以传统的Dubins方法不适用。本文考虑将该问题分为横向曲线设计和纵向曲线设计两部分。

横向曲线设计忽略起始航点的高度差,将问题转化为传统的Dubins轨迹设计问题。定义离开初始航点虚拟圆的角度为$\psi_s$,进入结束航点虚拟圆的角度为$\psi_f$。
\begin{equation}
\psi_s = \left\{ \begin{array}{ll}
\text{atan}(Y_s - Y_{cs}, X_s-X_{cs}) ,&\mbox{if}\ P_s =P_{s,exit} \\
\text{atan}(Y_{f,entry} - Y_{cf}, X_{f,entry}-X_{cf}) ,&\mbox{ otherwise}
\end{array} \right.
\end{equation}

\begin{equation}
\psi_f = \left\{ \begin{array}{ll}
\text{atan}(Y_{s,exit} - Y_{cs}, X_{s,exit}  -X_{cs}) ,& \text{if}\ P_f = P_{f,entry}\\
\text{atan}(Y_{f} - Y_{cf}, X_{f}-X_{cf}) ,&\mbox{ otherwise}
\end{array} \right.
\end{equation}

在已知上述两个角度后,可以求得无人机航向角所需要旋转的最大角度
\begin{equation}
\psi_{max} = \left\{ \begin{array}{ll}
-|\psi_f-\psi_s|, & \text{if counter clockwise,}\ \psi_f - \psi_s \le 0\\  
-(2\pi-|\psi_f - \psi_s|),& \text{if counter clockwise,}\ \psi_f -  \psi_s > 0\\
|\psi_f-\psi_s|, & \text{if clockwise,}\ \psi_f - \psi_s \ge 0\\
(2\pi - |\psi_f - \psi_s|) ,& \text{if clockwise,}\ \psi_f - \psi_s < 0\\
\end{array} 
\right.
\end{equation}
求解无人机最大旋转角度的目的是为了计算切割曲线的角度步长$S_{angle}$和航迹点总步长数$N$。在定义最小离散航点的距离$S_{min}$之后,角度步长和航迹点总步长数量的计算公式如下
\begin{align}
S_{angle} = \text{sign} (\psi_{max}) \frac{S_{min}}{R} \\
N = \lceil \frac{|\psi_{max}|}{|S_{angle}|} \rceil+1
\end{align}
由上面两个步长,可以求解出在曲线位置和直线位置的离散航迹点序列。每一个航迹点的航向角序列为
\begin{equation}
\psi(\bar{\lambda}) = \left\{ \begin{array}{ll}
\psi_{max}, &  \bar{\lambda}=N-1\\
\bar{\lambda}S_{angle}, &\mbox{ otherwise}
\end{array} \right.
\end{equation}
其中$\bar{\lambda} = 1, ..., N-1$是一个连续序列。基于航向角离散序列,可以将每一个离散航迹点重新定义
\begin{equation}
\mathbf{WP}_{i} = \left\{ \begin{array}{ll}
(X_{cs}\ Y_{cs}\  0)^T +R_s (\cos (\psi_s+\psi(\bar{\lambda}))\ \sin(\psi_s+\psi(\bar{\lambda}) \ 0)^T& \text{if}\ 1< i \le N_s\\
(X_{cs}\ Y_{cs}\ 0)^T + R_f(\cos(\psi_s+\psi(\bar{\lambda})\ \sin(\psi_s+\psi(\bar{\lambda}) \ 0)^T, & \text{if}\ N_s< i \le N_f
\end{array} \right.
\end{equation}
其中$i \in [1, ..., N_s+N_f]$。根据上述公式求解得到的Dubins轨迹如图\ref{fig:chp07_06_dubins_approach_path}所示。
\begin{figure}[ht]   
	\centering
	\includegraphics[width=\textwidth]{figs/chp07/chp07_06_dubins_approach_path.pdf}
	\caption{二维平面Dubins曲线规划结果示意图}
	\label{fig:chp07_06_dubins_approach_path}
\end{figure}

在得到每个航点前两项后,需要根据飞机下降的情况计算每个航点对应的高度。高度下降的表达可以通过下滑角来描述,每一个航点的下滑角根据航点个数进行均匀分配。
\begin{equation}
z_i^{path} = \sqrt{(x_i^{path}-x_{i-1}^{path})^2+(y_i^{path}-y_{i-1}^{path})^2} \tan (\gamma_{d,i})
\end{equation}
其中$\gamma_{d,i}$定义为
\begin{equation}
\gamma_{d,i} = \text{atan}(z_{d}-z_i^{path}, \  \sqrt{(x_i^{path}-x_{i-1}^{path})^2+(y_i^{path}-y_{i-1}^{path})^2})
\end{equation}
其中$z_d$是上一节中第六航点的高度。图\ref{fig:chp07_07_dubins_approach_without_sprial}所示为将高度生成之后的进近段曲线与下降曲线合成的结果。
\begin{figure}[ht]   
	\centering
	\includegraphics[width=\textwidth]{figs/chp07/chp07_07_dubins_approach_without_sprial.pdf}
	\caption{无等距螺旋线的无人机降落曲线示意图}
	\label{fig:chp07_07_dubins_approach_without_sprial}
\end{figure}
在无人机初始位置与下降曲线初始位置的高度差过大时,上述算法的下滑角$\gamma_{d,i}$可能会超过最大下滑角阈值$\gamma_{max}$,所以在经过二维平面设计的Dubins曲线之后,仍然无法到达$P_f$点期望的高度位置。由此,考虑引入等距螺旋曲线来引导无人机在$P_f$附近盘旋,知道无人机逐渐下降到合适高度后,切入下降阶段航线。等距螺旋曲线每个点的下滑角不超过最大下滑角阈值、螺旋线的旋转半径与结束航点虚拟圆大小相同、旋转方向也与该虚拟圆轨迹保持相同。根据第二章的定义,下滑角最大约束为$\bar{\gamma}$,可以得到高度差的范围
\begin{equation}
(L_{dubins} + 2\pi k R_{min})) \tan \bar{\gamma} \le |z_d - z_s|\le (L_{dubins} + 2\pi(k+1)R_{min})) \tan \bar{\gamma}
\end{equation}
其中$L_{dubins}$是无人机在沿最小虚拟圆$R_{min}$飞行时所经过的轨迹长度。通过上式,可以求解出无人机需要经过等距螺旋线下降到期望高度所需要旋转周期数
\begin{equation}
k = \lfloor \frac{1}{2 \pi R_{min}} (\frac{|z_d - z_s|}{\tan \bar{\gamma}} - L_{dubins}) \rfloor
\end{equation}
引入螺旋下降曲线之后的完整下降轨迹如图\ref{fig:chp07_08_dubins_approach_with_sprial}所示。
\begin{figure}[ht]   
	\centering
	\includegraphics[width=\textwidth]{figs/chp07/chp07_08_dubins_approach_with_sprial.pdf}
	\caption{引入等距螺旋线的无人机降落曲线示意图}
	\label{fig:chp07_08_dubins_approach_with_sprial}
\end{figure}



\subsection{航迹点坐标变换}
为了便于计算和设计,上面两节设计的下降轨迹航迹点均是在降落轨迹局部坐标系得到。因此,需要通过旋转矩阵,将降落轨迹局部坐标系的航迹点转换到系统坐标系中,一般而言,降落轨迹局部坐标系的$\mathbf{i}^{path,l}$和$\mathbf{j}^{path,l}$构成的平面与系统坐标系$\mathbf{i}^{i}$和$\mathbf{j}^{i}$组成的平面平行。定义沿系统坐标系$\mathbf{k}^{i}$轴旋转$\psi_{path,l}$得到降落轨迹局部坐标系。由此,系统坐标系中的航迹点$\mathbf{WP}_n$与降落轨迹局部坐标系的航迹点$\mathbf{WP}_n^{path,l}$之间的转换关系为
\begin{align}
\mathbf{WP}_n^{path,l} = \mathbf{R}_i^{path,l}(\psi_{path}) \mathbf{WP}_n
\end{align}
其中$ \mathbf{R}_i^{path,l}(\psi_{path}) $ 是标准的旋转矩阵
\begin{align}
\mathcal{R}_i^{path,l}(\psi_{path}) = \begin{bmatrix} \cos \psi_{path}  & \sin \psi_{path}  &  0      \\  - \sin \psi_{path} & \cos \psi_{path}   &0 \\  0   & 0 &1  \end{bmatrix}
\end{align}
​
%
%\section{基于L1方法的导航控制策略}
%\subsection{基于直线情况}


%\begin{algorithm2e}[ht]
%	\SetAlgoLined
%	\KwData{this text}
%	\KwResult{how to write algorithm with \LaTeX2e }
%	initialization\;
%	\While{not at end of this document}{
%		read current\;
%		\eIf{understand}{
%			go to next section\;
%			current section becomes this one\;
%		}{
%			go back to the beginning of current section\;
%		}
%
%	}
%	\caption{How to write algorithms}
%\end{algorithm2e}
%
%\begin{algorithm2e}{ht}
%	\SetKwData{Left}{left}\SetKwData{This}{this}\SetKwData{Up}{up}
%	\SetKwFunction{Union}{Union}\SetKwFunction{FindCompress}{FindCompress}
%	\SetKwInOut{Input}{Input}\SetKwInOut{Output}{Output}
%	\Input{A bitmap $Im$ of size $w\times l$}
%	\Output{A partition of the bitmap}
%	\BlankLine
%	\emph{special treatment of the first line}\;
%	\For{$i\leftarrow 2$ \KwTo $l$}{
%		\emph{special treatment of the first element of line $i$}\;
%		\For{$j\leftarrow 2$ \KwTo $w$}{\label{forins}
%			\Left$\leftarrow$ \FindCompress{$Im[i,j-1]$}\;
%			\Up$\leftarrow$ \FindCompress{$Im[i-1,]$}\;
%			\This$\leftarrow$ \FindCompress{$Im[i,j]$}\;
%			\If(\tcp*[h]{O(\Left,\This)==1}){\Left compatible with \This}{\label{lt}
%				\lIf{\Left $<$ \This}{\Union{\Left,\This}}
%				\lElse{\Union{\This,\Left}}
%			}
%			\If(\tcp*[f]{O(\Up,\This)==1}){\Up compatible with \This}{\label{ut}
%				\lIf{\Up $<$ \This}{\Union{\Up,\This}}
%				\tcp{\This is put under \Up to keep tree as flat as possible}\label{cmt}
%				\lElse{\Union{\This,\Up}}\tcp*[h]{\This linked to \Up}\label{lelse}
%			}
%		}
%		\lForEach{element $e$ of the line $i$}{\FindCompress{p}}
%	}
%	\caption{disjoint decomposition}\label{algo_disjdecomp}
% 	\caption{使用SCP方法求解侧风扰动下的最优轨迹}
% \end{algorithm2e}

\section{舰船与无人机协同速度控制策略}
\subsection{协同速度控制策略}
假设舰船的初始速度为零,无人机从舰船后方窗口进入下降轨迹。在初始加速阶段,认为无人机的速度保持不变,舰船的加速度保持不变,由此可以得到如下表达
\begin{align}
V_{Ship} = V_{UAV} - \sqrt{2 \cdot a_{Ship} \cdot (x_{Ship} - x_{UAV})}
\end{align}
其中无人机的速度为$V_{UAV}$,舰船的加速度为$a_{Ship}$。

由于地面移动平台的初始速率$V_{Ship}=0$,将上式进行变换后,得到无人机与地面移动平台的相对位置为
\begin{align}
x_{Ship}-x_{UAV} = \frac{V_{UAV}^2}{2 \cdot a_{Ship}}
\end{align}
无人机降落过程中的速度一般为$19\ m/s$,地面运动平台加速度一般为最大加速度的一半,大约为$0.95\ m^2/s$,因此可以得到相对位置为$190\ m$。

无人机在降落过程中需要将控制空速的指令转换为控制地速的指令,由此可以减少降落过程中空气扰动的影响,位置相对稳定的地速,完成对地面移动平台的跟踪。
\begin{align}
V_{A_{des}} = (V_{k_{des}} -V_k) + V_A
\end{align}
其中$V_{k_{des}}$是期望地速(Ground Speed),空速误差$e_{V_A}$定义为
\begin{align}
e_{V_{A}} &= V_{A_{des}}-V_A \\
&= V_{k_{des}} - V_k \\
&= e_{V_k}
\end{align}
通过协同控制器的设计思路,考虑无人机与舰船之间位置的影响,无人机期望地速的表达方式为
\begin{align}
V_{k_{des}} = V_{k_{land}} + k \cdot (x_{Ship} - x_{UAV})
\end{align}
其中$V_{k_{land}}$是无人机地速控制指令,可以通过下述公式表达
\begin{align}
V_{k_{land}} &= V_{A_{land}} + \overline{V_k - V_A} \\
&= V_{A_{land}} + V_\omega
\end{align}
其中$V_\omega$是描述无人机空速和地速的差,$V_\omega$的获得通过一个低通滤波器,其截断频率为$0.2\ \text{Hz}$ 得到。

\subsection{地速和下降速率计算}
根据上述推导,可以得到无人机地速控制指令为
\begin{align}
V_{A_{des}} =( V_{A_{land}} + k \cdot (U_{Ship} - U_{UAV})) + V_\omega - V_k + V_A
\end{align}
无人机竖直方向跟踪方法为
\begin{align}
&\chi_{UAV} = \chi_{ll} + k_1 \cdot (y_{Ship} - y_{UAV}) +k_2 \frac{1}{s} (y_{Ship} - y_{UAV}) + k_3 (v_{Ship} - v_{UAV}) \\
&\chi_{Ship} = \chi_{ll} + k_4 \cdot (y_{Ship} - y_{UAV}) +  k_5 (v_{Ship} - v_{UAV})
\end{align}

纵向控制方法中下降速率的计算为
\begin{align}
\dot{h}_{flare} = k \cdot (h + h_B)
\end{align}
在考虑到无人机地速的情况上式变为
\begin{align}
\dot{h}_{flare} = \frac{V_{UAV}}{T_o V_{ref}} \cdot (h + h_B)
\end{align}
其中$V_{UAV}$是无人机的地速,$V_{ref}$是参考速度,$T_0$和$h_B$是下滑率设计的系统参数。

其中常数$k$ 和 $h_B$需要满足一些约束
\begin{align}
&\dot{h}_{flare} (f_{flare}) = \dot{h}_{descent} \\
&\dot{h}_{descent}(0)  = 0
\end{align}
设置初始高度为$h_{flare} = 5\ m$,根据该约束,在到达期望着舰点时,相对速度为零。

​\section{TECS耦合控制器设计}
\subsection{总能量的基本定义}
总能量控制求出无人机单纯的抬头动作(Nose-up),不仅会提升无人机的高度,同时会降低无人机的速度。图\ref{fig:chp07_14_Control_Response}所示为无人机升降舵和推力控制对无人机的高度和速度带来的影响,其中的中心点表示无人机的平衡点(Trim Point)。

\begin{figure}[ht]   
	\centering
	\includegraphics[width=0.4\textwidth]{figs/chp07/chp07_14_Control_Response.pdf}
	\caption{推力和升降舵控制对速度和高度的耦合影响示意图}
	\label{fig:chp07_14_Control_Response}
\end{figure}

总能量控制中的能量主要指动能(Kinetic)和势能(Potential)之和。其中电机的推力变化对总能量的变化
\begin{align}
P_{thrust} = T \cdot V
\end{align}
其中$T$是电机的推力,$V$是无人机的速度。

阻力对总能量的变化
\begin{align}
P_{drat} = -D \cdot V
\end{align}
其中$D$是阻力的大小。因此总能量在单位时间变化的计算公式为
\begin{align}
P=(T-D)\cdot V
\end{align}
从能量的定义角度来看,无人机的总能量的表达为
\begin{align}
\label{eq:energy}
E = \frac{1}{2} mV^2 + mg(h - h_0)
\end{align}
由此可以得到总能量的变化率
\begin{align}
\label{eq:energy_dot}
\dot{E} = mV\dot{V} + mg\dot{h}
\end{align}
根据公式\ref{eq:energy}和\ref{eq:energy_dot},可以得到
\begin{align}
mV\dot{V} + mg\dot{h} =(T-D)\cdot V
\end{align}
将上述公式继续整理,可以得到一个没有量纲的表达
\begin{align}
\label{eq:energy_3}
\frac{\dot{V}}{g} + \frac{\dot{h}}{V}  =\frac{T-D}{mg}
\end{align}

\subsection{内环设计和外环设计化简}
根据无人机运动的集合关系
\begin{align}
\frac{\dot{h}}{V} = \sin \gamma
\end{align}
由于无人机降落过程中,下滑角的大小一般在$3\degree \sim 7\degree$,因此$\sin \gamma \approx \gamma$,可以得到公式\ref{eq:energy_3}的进一步化简
\begin{align}
\frac{\dot{V}}{g} + \gamma =\frac{T-D}{mg}
\end{align}
考虑到阻力的变化一般比较小,因此可以用$T-D\approx \delta T$近似,可以得到
\begin{align}
\frac{\dot{V}}{g} + \gamma = \frac{\delta T}{mg}
\end{align}
由上式可以看到,系统通过电机推力的变化对能量的变化率进行控制。左侧两项的计算可以通过升降舵的控制来实现,因为升降舵的控制并不影响总能量的变化率。一般而言,升降舵的控制影响攻角的变化,与无人机加速度的变化成正比,与下滑角的变化成反比,因此控制量一般用$\frac{\dot{V}}{g} - \gamma $进行表达。

在实际系统中,TECS方法的s函数表达为
\begin{align}
&\frac{T_{des}}{\mathsf{m}g} = \frac{K_{TI}}{s} (\gamma_{des} - \gamma + \frac{\dot{V_{des}}}{g} - \frac{\dot{V}}{g}) - K_{TP}(\gamma + \frac{ \dot{V}}{g}) \\
&\theta_{des} = \frac{K_{EI}}{s} (\gamma_{des} - \gamma - \frac{\dot{V_{des}}}{g} + \frac{\dot{V}}{g}) - K_{EP}(\gamma - \frac{ \dot{V}}{g})
\end{align}
上式中的控制方法主要通过比例-积分(PI)方法实现,其中比例部分并没有与期望下滑角之差进行比较。由此可以得到,TECS的输出为期望俯仰角$\theta_{des}$和无量纲的期望推力$T_{des}/\mathsf{m}g$。

内环控制器主要解决无人机对于TECS的两个输出量的控制。对于期望俯仰角的控制主要通过无人机的升降舵来实现
\begin{align}
\delta_e = -(K_\theta+\frac{K_{i\theta}}{s})(\theta_{des}-\theta)-K_q(q_{turn}-q)
\end{align}
其中$q_{turn}$是完成协同转弯的指令,其具体表达为
\begin{align}
q_{turn} = \sin\phi \cos\theta\tan \phi \frac{g}{V_k}
\end{align}
此时无人机推力的控制主要通过电机转速的控制来实现
\begin{align}
\delta_t = (K_T + \frac{K_{iT}}{s})(T_{des} - T) + \delta_{tFF}
\end{align}

外环设计主要是TECS模块的输入,即$\gamma$和$\dot{V}$两个控制量。此外,还困成通过与期望值的之差的比例控制方法得到
\begin{align}
&\gamma_{des} = \frac{\dot{h}_{des}}{V_k} \\
&\dot{V}_{des} = K_V(V_{des}-V_a) 
\end{align}
其中$\dot{h}_{des} = K_h(h_{des}-h)$。无人机的飞行特性存在一定的约束,因此根据总能量公式,期望数值必须满足如下约束
\begin{align}
&\dot{V} _{des} \le -\frac{Dg}{\mathsf{m}g} \\
&\gamma_{des} \le \frac{T_{max} - D}{\mathsf{m}g}
\end{align}

由于一般无人机无法测量侧滑角,所以横向控制器的设计主要通过对航向角的变化率进行控制,输出值为
\begin{align}
&\delta_r = -K_r (r_{turn - r}) \\
&r_{turn} = \cos \phi \cos \theta \tan \phi \frac{g}{V_k}
\end{align}

\section{MPC控制方法}

\subsection{侧风扰动情况下的轨迹规划方法}
在许多开源飞行控制器中,对于轨迹的跟踪通常采用L1引导控制方法,该方法实质上类似于比例-微分控制器(PD Controller)。但如果考虑在不同海况情况下的风速的影响,无人机实际上很难正确跟踪上一节生成的Dubins曲线。因此,可以通过引入水平方向的扰动,进一步分析问题,并使用最优化的方法进行求解。该方法实质上是一种模型预测控制(Model Predictive Control,MPC)。无人机的优化目标为从初始位置到达终止位置的所需时间$T$最短,其数学表达为 
\begin{align}
\textsf{minimize}\ \ T \label{eq:gust_dubins_1} \\
\textsf{subject to}\  \ &\dot{\mathbf{p}}(t) = {V}_a(\cos \psi(t)\ \ \sin \psi(t))^T + \mathbf{\omega}  \nonumber \\
& \dot{\psi}(t) = \frac{g}{{V}_a} \tan \phi(t) \nonumber \\
& \dot{\phi}(t) = u(t) \nonumber \\
& \mathbf{p}(0) = \mathbf{p}_{start} \nonumber \\
& \psi(0) = \psi_{start} \nonumber \\
& \phi(0) = \phi_{start} \nonumber \\
& \mathbf{p}(T) = \mathbf{p}_{final} \nonumber \\
& \psi(T) = \psi_{final} \nonumber \\
& \phi(T) = \phi_{final} \nonumber \\
& |u(t)| \le \bar{u}  \nonumber \\
& |\phi(t)| \le \bar{\phi} \nonumber \\
& x(y), y(t) \in \mathbb{C} \nonumber \\
& t \in [0, T] \nonumber
\end{align}

其中无人机的空速仍然定义为$V_a$,为简化问题设定该段时间的风速为一个恒定值$\omega$。通过最优方法得到的轨迹用$x(t) = (\mathbf{p}(t)\ \ \psi(t)\ \ \phi(t)\ \ u(t) )^T$进行描述,其中$\mathbf{p}(t)$是无人机当前在惯性坐标系下的位置,$\psi(t)$是无人机的航向角,$\phi(t)$是无人机的横滚角,$u(t)$是无人机的横滚角速度控制量。此外,无人机的初始位置定义为$x_{start}=x(0)$,终止位置为$x_{final}=x(T)$,该问题的几何表达如图\ref{fig:chp07_09_scp_path}所示。由于无人机自身模型和执行机构的约束,设定横滚角和最大速度输出为$\bar{\phi}$和$\bar{u}$。
\begin{figure}[ht]   
	\centering
	\includegraphics[width=\textwidth]{figs/chp07/chp07_09_scp_path.pdf}
	\caption{引入等距螺旋线的无人机降落曲线示意图}
	\label{fig:chp07_09_scp_path}
\end{figure}

分析上述公式可以发现,其约束条件中存在大量非线性部分,所以求解过程相对复杂,且无法满足无人机实时性需要,因此需要对相关函数进行线性化。无人机位置的变化率可以通过Taylor展开进行线性化,
\begin{align} {\dot{\mathbf{p}}(\phi\ \omega\ t) \approx {V}_a \begin{bmatrix} \cos \hat{\psi}(t)^{(i)} - \sin\hat{\psi}(t)^{(i)} ({\psi}(t)^{(i)} - \hat{\psi}(t)^{(i)}) \\ \sin \hat{\psi}(t)^{(i)}+ \cos \hat{\psi}(t)^{(i)}({\psi}(t)^{(i)} - \hat{\psi}(t)^{(i)})  \end{bmatrix} + \omega}
\end{align}
其中$\hat{\psi}(t)^{(i)}$表示沿轨迹各个点估计的航向角。

横滚角速率可以通过如下公式近似
\begin{align}
\dot{\psi}(t) \approx \frac{g}{V_a} \phi(t)
\end{align}

连续曲线生成之后,需要设计方法将上述曲线离散化。在时间段$t \in [0, T]$之间,一般使用$N$个离散点将期望的轨迹长度$L$进行分割,由此得到每一个时间间隔为
\begin{align}
\Delta s_{path} = \frac{T}{N}
\end{align}
由此,当前时刻可以用$t =k \Delta s_{path}$表达,则每个航迹点变化率可以写成相邻离散点之差的形式
\begin{align}
\dot{\mathbf{p}}(t) &\approx \dot{\mathbf{p}}(k \Delta s_{path}) \\
&= \frac{\mathbf{p}_{k+1}-\mathbf{p}_{k}}{\Delta s_{path}}
\end{align}
将上述公式线性化之后,选择以横滚角的一范数作为目标函数,由此可以得到拟最优化(Quasi-convex)问题
\begin{align}
\textsf{minimize}\ \ &||\phi|| \\
\textsf{subject to}\  \ x(k+1) &= \mathbf{A}(k, \hat{\psi}, \Delta s_{path}, {V}_a, \omega)x(k) + \mathbf{B}(k, \hat{\psi}, \Delta s_{path}, {V}_a) \nonumber  \\
|x(k)| & \preceq x_{max} \nonumber \\
x(0) &= x_{start} \nonumber \\
x(N) &= x_{final} \nonumber \\
k &= 1... N \nonumber
\end{align}
其中$x(k) = (\mathbf{p}(k)\ \ \psi(k)\ \ \phi(k)\ \ u(k))^T,\ \ k=1,2,...,N$表示无人机在第$k$时间段的状态,$\mathbf{A} \in \mathbb{R}^{5\times5}$和$\mathbf{B} \in \mathbb{R}^{5\times1}$是状态迭代过程中的状态转换矩阵。根据序列凸优化(Sequential Convex Programming)方法,此时需要对线性化之后的$\hat\psi$设定一个置信区域(Trust Region),该区域的表达为
\begin{align}
\psi \le \rho \in \mathcal{T}
\end{align}
SCP方法主要是通过迭代更新$\hat{\psi}(k)$来计算局部最优轨迹,该轨迹满足原始非凸优化问题的约束,公式\ref{eq:gust_dubins_1}。$\hat{\psi}(k)$本质上就是每个离散航点位置无人机的航向角,由此可以得到无人机状态量的迭代公式
\begin{align}
\mathbf{p}(k+1) - \mathbf{p}(k)  &= V_a \Delta S_{path} \begin{bmatrix} \sin \hat{\psi}(k) \\ \cos\hat{\psi}(k)   \end{bmatrix} \\
\dot{\mathbf{p}}(k+1) - \dot{\mathbf{p}}(k) &=  \Delta S_{path} (V_a \begin{bmatrix} \cos \hat{\psi}(k) + \sin \hat{\psi}(k)\cdot \hat{\psi}(k) \\ \sin \hat{\psi}(k) - \cos \hat{\psi}(k)\cdot \hat{\psi}(k)  \end{bmatrix} + w) \\
\phi(k+1) - \phi(k) &=   \Delta S_{path}  \frac{g}{V_a} 
\end{align}

综上所示,可以算法基本步骤如下:
\begin{compactenum}
	\item 求解序列初始化:与传统最优化方法随机产生初始值不同,$\hat{\psi}(k)$序列的初始化是通过简单的线性迭代方法得到,其初始值和终止值分别为$\psi(0)$和$\psi(N)$。具体到本文的路径规划问题,首先通过主要无人机的初始位置和终止位置进行直线连接,随后根据无人机的运动约束得到离散之后各个航点的位置和航向角,这些航向角作为$\hat{\psi}(k)$序列的初始值。
	\item 置信区域初始化:一般而言初始化区域是围绕在当前点周边的区域,在序列求解过程中的序列表达为
	\begin{align}
	\mathcal{T}(k) = \{\psi(k)|\ |\hat{\psi}(k)-{\psi}(k)| \le \rho(k),\ k=1,2,...,N \}
	\end{align}
	置信区间在迭代过程根据误差值进行缩放,通常引入缩放因子$\gamma_{trust}$进行进行描述。
	
	\item 最优结果的输出:通过迭代,最后得到在给定初始状态和约束条件下每一步的无人机状态$\hat{x}(t)^{(i)}$。
	
	\item 迭代终止条件:设定状态量各个分量的线性误差$x_{error} \le \alpha$ 且$|x^{*(i)} - x^{*(i-1)} | \le \beta$时终止迭代。
	
	\item 原始非凸最优问题的系统输入计算:在每一步迭代过程中,将单步最优解$\hat{x}^{*}(k)$与$x(k)$进行比较,即可估计出系统的期望输入$\hat{u}(k)$,该输入即为原始非凸最优问题的输入。
\end{compactenum}
在代码实现的过程中,SCP问题的求解主要依赖CVXGEN\cite{mattingley2012cvxgen}计算进行求解。综上所述,在侧方扰动下的最优轨迹求解算法如算法\ref{alg:gust_scp}所示。

\begin{algorithm2e}
	\SetAlgoLined
	\SetKwInOut{Input}{Input}
	\SetKwInOut{Output}{Output}
	\Input{Initialized sequence $\hat{\psi}(k),\ \ k=1,2,..., N$}
	
	\KwResult{$x(k)$}
	%initialization\;
	\While{$\Sigma |x_{error}| \le \alpha$ and $|x^{*(i)} - x^{*(i-1)} | \le \beta$}
	{
		Update UAV states \;
		$\mathbf{p}(k+1) \leftarrow$ Update($\mathbf{p}(k)$)\;
		$\dot{\mathbf{p}}(k+1) \leftarrow$ Update($\dot{\mathbf{p}}(k)$)\;
		${\phi}(k+1) \leftarrow$ Update(${\phi}(k)$)\;
		Solve SCP using CVXGEN Solver\;
		\For{$j\leftarrow 1$ \KwTo $N-1$}{
			Calculate $x_{error}$ \;
		}
		Update $\hat{\psi}(k)^{i+1} = \hat{\psi}(k)^{i} $\;
		Update Trust Region, $\rho(k)^{i+1} = \gamma_{path}(k)^{i} \rho(k)^{i}$ \;
		$i = i + 1$
	}
	\caption{侧风扰动下的最优轨迹求解}
	\label{alg:gust_scp}
\end{algorithm2e}

由于无人机降落方向一般是逆风飞行,因此在进入降落曲线的进近阶段无人机周边的风速一般以顺风或正侧风。由此,根据上述求解方法,设定初始航点和终止航点在降落曲线局部坐标系的坐标分别为$p_{start} = (0,0)$和$p_{final}=(140,100)$,初始航点航向角和终止航点航向角为$\psi_{start}=180\degree$和$\psi_{final} = -90\degree$,SCP算法的置信区间为$180\degree$。在风速为$3\ m/s$,风向为$0\degree$的侧风扰动情况下,Dubins路径规划方法、普通MPC路径规划方法和考虑侧风影响情况下的MPC路径规划方法的运算结果如图\ref{fig:chp07_12_three_path}所示。图中的深蓝色实线为传统Dubins方法在二维空间对起始点和终止点的路径规划,绿色点画线和紫色曲线分别是MPC方法在不考虑侧风扰动和考虑侧风扰动情况下的曲线规划,浅蓝色箭头为方向指示。此时Dubins路径的长度为$282.25\ m$,MPC无侧风扰动的路径长度为$314.94\ m$,MPC有侧风扰动的路径长度为$323.25\ m$。MPC两种方法的横滚角、输入控制量和地速的比较如图\ref{fig:chp07_10_compare_mdp_path}所示,使用CVXGEN迭代过程的比较如图\ref{fig:chp07_11_mdp_path_compute}所示。在风速为$3\ m/s$,风向为$135\degree$的侧风扰动情况下,Dubins路径规划方法、普通MPC路径规划方法和考虑侧风影响情况下的MPC路径规划方法的运算结果如图\ref{fig:chp07_12_three_path}所示,此时路径长度为$343.57\ m$。基于CVXGEN库求解上述最优问题的平均运行时间为$0.048\ s$,约$20\ Hz$。

\begin{figure}[ht]   
	\centering
	\includegraphics[width=0.7\textwidth]{figs/chp07/chp07_12_three_path.pdf}
	\caption{在风向为0$\degree$情况下三种曲线生成方式的比较}
	\label{fig:chp07_12_three_path}
\end{figure}

\begin{figure}[ht]   
	\centering
	\includegraphics[width=0.8\textwidth]{figs/chp07/chp07_10_compare_mdp_path.pdf}
	\caption{在有侧风和无侧风情况下的横滚角、控制量和地速的比较}
	\label{fig:chp07_10_compare_mdp_path}
\end{figure}

\begin{figure}[ht]   
	\centering
	\includegraphics[width=0.6\textwidth]{figs/chp07/chp07_11_mdp_path_compute.pdf}
	\caption{在有侧风和无侧风情况下的迭代步数的比较}
	\label{fig:chp07_11_mdp_path_compute}
\end{figure}



\begin{figure}[ht]   
	\centering
	\includegraphics[width=0.7\textwidth]{figs/chp07/chp07_13_45_three_path.pdf}
	\caption{在风向为135$\degree$情况下三种曲线生成方式的比较}
	\label{fig:chp07_13_45_three_path}
\end{figure}

\subsection{基于横向控制的模型预测控制策略}

上一节介绍的方法虽然能够通过模型预测控制的方法给出了在侧风扰动情况下的控制量求解方法,但由于无人机底层控制回路的实时性要求高,上述方法约$20\ Hz$的控制频率往往不能满足无人机姿态的控制需要。因此,需要考虑将底层控制与顶层控制独立设计的控制框架。无人机的控制器设计主要采用级联控制的设计思路(Casade Control),其中内环控制主要为底层控制(Low-level Control),外环控制是基于模型的轨迹跟踪控制器。该设计思路的具体实现如图\ref{fig:chp07_15_control_diagram}所示。

\begin{figure}[ht]   
	\centering
	\includegraphics[width=\textwidth]{figs/chp07/chp07_15_control_diagram.pdf}
	\caption{底层控制和顶层控制框图}
	\label{fig:chp07_15_control_diagram}
\end{figure}

由于使用Pixhawk作为无人机的底层控制器,其内环控制实现是基于修正后的PID级联控制结构。对于姿态控制,采用PI控制方法;对于角速度控制回路,采用PD控制方法。其中考虑到无人机在进行协调转弯过程中的侧滑影响(Slipping Effect),则航向角的控制指令为
\begin{align}
r_r=\frac{g\sin\phi}{V}
\end{align}
其中TECS来完成对系统的速度和高度控制。

为便于问题分析,假设无人机机体坐标系与风向坐标系重合,根据第二章的无人机机体坐标系定义可以得到化简的无人机动力学公式
\begin{align}
&\dot{n}(t) = V\cos \psi(t) + w_n \\
&\dot{e}(t) = V\sin \psi(t) + w_e \\
&\dot{\psi}(t) = \frac{g}{V}\tan\phi(t) \\
&\dot{\phi}(t) = p(t) \\
&\dot{p}(t) = b_0\phi_{des}-a_1p-a_0\phi \label{eq:roll_s_function}
\end{align}
其中$\dot{n}$和$\dot{e}$表示无人机在惯性坐标系的位置变化量,$a_0$、$a_1$和$b_0$是无人机模型的系数。其他数学符号的含义与第二章的相同。如公式所示,选用没有零点的二阶模型来描述无人机的横滚运动。

考虑到实际实验中使用的Odroid XU4的综合运算能力仍然有限,因此只对滚转控制通道进行模型预测控制。在底层控制器工作相对理想的情况下,需要在俯仰角控制指令不变的情况下,对滚转通道进行辨识。参考信号的一般使用2-1-1方波信号。文献\cite{morelli2003low}在设计Tu-144超音速飞行器的过程中发现,该类型的激励信号能够在飞行时间有限的情况下,尽可能多的采集到相应数据,相比传统的频率切换激励方法,该方法的测试时间仅为传统方法的1/6。因此,使用连续的2-1-1信号可以满足不同频域和时域的系统辨识。

在实际系统参数辨识的过程中,首先通过对Pixhawk的底层控制PID参数进行调节。在底层控制器对姿态达到较好的控制品质的情况下,通过对横滚通道输入2-1-1参考信号,从而实现对该通道参数的辨识。因此可以通过在实际飞行中对底层控制器给出期望的滚转角控制信号$\phi_{der}$,通过传感器测得的$\phi$,即可计算出横滚角通道的传递函数$\phi / \phi_{des}(s)$。对于横滚通道而言,一般使用频率为$1\ Hz$,赋值为$30 \degree$的方波。通过使用Matlab/Simulink来搭建仿真环境来进行横滚通道的辨识和验证,该仿真系统如图\ref{fig:chp07_17_system_identification}所示。通过图中左下侧的参考信号生成模块,在高度和速度保持不变的情况下,横滚角通道的期望输入和实际输出如图\ref{fig:chp08_18_roll_roll_desired_compare}所示。

\begin{figure}[ht]   
	\centering
	\includegraphics[width=\textwidth]{figs/chp07/chp07_17_system_identification.pdf}
	\caption{横滚角通道系统识别仿真系统}
	\label{fig:chp07_17_system_identification}
\end{figure}

\begin{figure}[ht]   
	\centering
	\includegraphics[width=\textwidth]{figs/chp07/chp08_18_roll_roll_desired_compare.pdf}
	\caption{使用2-1-1信号激励横滚角通道的实验结果}
	\label{fig:chp08_18_roll_roll_desired_compare}
\end{figure}

在得到上述实验数据之后,通过设计一个二阶传递函数来得到\ref{eq:roll_s_function}中的参数值。该二阶传递函数为
\begin{equation}
\frac{b_0}{s^2+a_1s+a_0}
\end{equation}

在得到滚转通道的系统模型之后,需要进一步求解非线性模型预测控制的方程。对于连续时间的最优问题,其数学表达形式为
\begin{align}
\min_{UX} \int_{t=0}^T &(y(t)-y_{ref}(t))^TQ_y(y(t)-y_{ref}(t))+(u(t)-u_{ref}(t))^T R_u (u(t)-u_{ref}(t))dt\\
&+ (y(T)-y_{ref}(T))^T P (y(T)-y_{ref}(T))
\end{align}
\begin{align}
\textsf{subject to}\  &\dot{x}=f(x,u) \\
&y=h(x,u) \\
&u(t) \in \mathbb{U} \\
&x(0) = x(t_0)
\end{align}
其中系统的状态向量为$x=[n,e,\phi,\psi,p]$,系统的输出为滚转角控制指令$u=\phi_{des}$。

对于路径跟踪问题,其控制目标之一是最小化距离,该距离定义为
\begin{align}
e_t = (\mathbf{d}-\mathbf{p}) \times \mathbf{T}_d
\end{align}
其中$\mathbf{T}_d$是当前无人机位置距离跟踪曲线最近点$\mathbf{d}$的切线方向,$\mathbf{p}$为无人机的当前位置。另一个控制目标是航迹航向角与当前点的方位角之差同时变小,该误差定义为
\begin{align}
e_{\chi} = \chi_d-\chi
\end{align}
其中$\chi_d$可以通过以下公式求解
\begin{align}
\chi_d = \text{atan} (T_{d_e}, T_{d_n}) \in [-\pi, \pi]
\end{align}
由此可以得到目标控制量为$y=[e_t,e_\chi,\phi,p,\phi_r]$。上述变量的几何关系如图\ref{fig:chp07_16_path_design}所示。

\begin{figure}[ht]   
	\centering
	\includegraphics[width=\textwidth]{figs/chp07/chp07_16_path_design.pdf}
	\caption{轨迹设计几何关系示意图}
	\label{fig:chp07_16_path_design}
\end{figure}


为了表述方便,将Dubins曲线中的直线部分和结束虚拟圆部分用数学参数表达,其中直线段的数学公式为
\begin{align}
\mathcal{P}_{line} = \{type=0,\mathbf{P}_{s,exit},\mathbf{P}_{f,entry}\}
\end{align}
其中$\mathbf{P}_{s,exit}$和$\mathbf{P}_{f,entry}$定义了Dubins曲线中的直线段的起始点。Dubins曲线中的结束虚拟圆的数学表达为
\begin{align}
\mathcal{P}_{arc} = \{type=1,\mathbf{O}_{cf},R_f,direction,\psi_f\}
\end{align}
其中$\mathbf{O}_{cf}$是Dubins曲线结束位置虚拟圆的圆心,该圆的半径为$R_f$,$direction$是旋转方向。
通过引导系统给出的无人机当前位置,在满足Dubins Path切换条件的情况下,实时期望跟踪曲线的转换。

\section{无人机降落决策设计}
一般而言,无人机的最终决断时间在接触甲板之前的1.5s至12.5s。定义下滑曲线的方位角误差为$\delta_h$,距离跑道中心线的误差为$\epsilon_h$,下滑曲线的俯仰角误差为$\delta_v$,距离期望下降曲线的高度差为$\epsilon_v$。无人机降落决策系统主要是通过上述多传感器的解算结果,在决断窗口根据无人机下降阶段的六个基本数据做出是否复飞的判断。这六个量分别为:
\begin{compactenum}
	\item 跑道中心线的误差为$\epsilon_h$
	\item 距离期望下降曲线的高度差为$\epsilon_v$
	\item 无人机横滚角$\phi$
	\item 无人机俯仰角$\theta$
	\item 航母甲板下沉速率$\dot{h_c}$
	\item 航母甲板侧滑速率$v_s$
\end{compactenum}

根据上述六个不同的基本数据指标,无人机主要面临的六种危险如下表所示。

\begin{table}[ht]
	\centering
	\caption{无人机降落过程中的主要危险}
	\begin{tabular}{cccc}
		\hline
		\textbf{风险名称} & \textbf{符号表达} & \textbf{相关变量} & \textbf{状态描述}       \\ \hline
		撞击甲板下方 & PRS  & $e_v$ & 超过纵向最大负阈值  \\
		未接触拦阻索 & PUL  & $e_v$ & 超过纵向最大正阈值  \\
		横向偏差大  & POL  & 无   & 超过横向最大正负阈值 \\
		姿态偏差大  & PBA  & 无    & 着舰姿态偏差大    \\
		硬着舰    & PHL  & 无   & 超过沉降最大阈值   \\
		横向偏离跑道 & PRB  & 无    & 无          \\  \hline
	\end{tabular}
\end{table}

其中$P_{RS}$定义为
\begin{equation}
	P_{RS}=\frac{1}{\sigma_v\sqrt{2\pi}}\int_{-\infty}^{e_{v,low}}\exp(-\frac{(x-\bar{e}_v)^2}{2\sigma_v^2})dx \\
	=\frac{1}{2}(1+\text{erf}(\frac{e_{v, low}-\bar{e}_v}{\sigma_v\sqrt{2}}))
\end{equation}
其中误差函数定义为
\begin{equation}
	\text{erf}(x)=\frac{2}{\sqrt{\pi}}\int_{0}^{x}e^{-t^2}dt
\end{equation}
假设六种情况发生的概率是相互独立的,由此定义无人机正常回收、复飞和失败的概率。上述误差量如图\ref{fig:chp06_05_landing_error}和图\ref{fig:chp06_06_landing_error_three_axis}所示。

\begin{align}
	&P_{recover}=(1-P_{RS})(1-P_{UL})(1-P_{OL})(1-P_{BA})(1-P_{HL})(1-P_{RB}) \\
	&P_{bolter}=(1-P_{RS})P_{UL}(1-P_{OL})(1-P_{BA})(1-P_{HL})(1-P_{RB}) \\
	&P_{fail}=1-(1-P_{RS})(1-P_{OL})(1-P_{BA})(1-P_{HL})(1-P_{RB})
\end{align}




\begin{figure}[!tb]
	\centering
	\includegraphics[width=\textwidth]{figs/chp06/chp06_05_landing_error.pdf}	
	\caption{降落过程中的下滑轨迹误差}
	\label{fig:chp06_05_landing_error}
\end{figure}

\begin{figure}[!tb]
	\centering
	\includegraphics[width=\textwidth]{figs/chp06/chp06_06_landing_error_three_axis.pdf}	
	\caption{降落过程中的误差分布情况}
	\label{fig:chp06_06_landing_error_three_axis}
\end{figure}

\section{本章小结}
本章主要针无人机的指导与控制系统进行设计。首先针对无人机的降落曲线进行设计,通过使用Dubins Path设计方法,针对无人机下降过程中的需要,设计了满足实际需要的无人机的下降曲线和进阶曲线。与传统的无人机在地面回收不同,无人机在着舰过程中,需要考虑无人机与舰船速度的相互关系,为此专门设计了协同速度控制策略。考虑到无人机着舰过程中容易受到侧风等环境扰动的影响,在平衡机载设备运算能力的情况下,设计了基于总能量控制和模型预测控制的外环控制策略与基于PD和PID控制的内环控制策略。

