
%%% Local Variables:
%%% mode: latex
%%% TeX-master: "../main"
%%% End:

\begin{ack}
在提笔书写这一段文字的时候,意味着读书生涯即将画上句号。十年求学是一段砥砺成长的旅程,我在这段旅程中的蜕变得益于身边众多老师、同学和朋友们的支持和帮助,得益于实验室、学院、学校以及这个时代的馈赠。

首先向我的导师沈林成教授致以深深的谢意和崇高的敬意!从硕士到博士,从选课到选题,从工作到生活,我的每一次进步都离不开您的悉心指导和辛勤付出。在您营造的自由开放、严谨务实的学术环境中,我的自主创新能力、独立思考能力和工程实践能力得到了很好的锻炼。您善于发现学生的优点,鼓励学生自我挑战,给予学生充分信任。早在硕士期间,我就有机会参加学校的创新杯和全国无人机大赛,在参与这些竞赛活动的过程中,我不仅收获了奖杯和荣誉,更让我对无人机领域有了真切的理解,为后面研究工作的顺利进行奠定了坚实基础。读博期间,在您的帮助和指导下,我再次获得机会到德国进行为期两年的联合培养,让我拥有更大的平台和更为广阔的视野。师恩如山,永生难忘!

感谢张代兵老师在硕博连读阶段的倾心指导。张老师给予的最大帮助在于培养我用一颗勇敢而坚韧的心去面对工程实践和日常生活中的重重困难。与张老师并肩工作的场景不是在熟悉的实验室,而是在深夜的长沙月亮岛、清晨的吉安、初冬的湘江岸边等外场。持续时间较长、条件相对艰苦的外场试验让我体会到理论推导、算法仿真到工程时间之路的艰辛。在学生时代,能够拥有这种并不顺畅的科研工作体会,是未来面对各类挑战的宝贵财富。

感谢胡天江老师在硕博连读阶段对我的帮助和指导。在修改会议论文、期刊论文和大论文的过程中,您的督促和指导以及无私奉献的精神让我感动。在读博的最后两年,您让我在ROS团队中承担更多的任务,协助带领团队开展相关工作,给予我充分的时间和空间来锻炼分析判断能力和组织领导能力。能够共同面对未知和不确定,共同努力攻关和协同奋进是一段美好的回忆。

%感谢徐晓红老师的严格要求和

感谢王祥科老师、张纪阳老师、相晓嘉老师、尹栋老师、周典乐老师、方强老师,在论文的开题和研究中给予的帮助,已经记不太清有多少次与您们讨论学术问题、编程方法和试验方案,这些点滴的讨论和细微的帮助,让我在学术研究过程中稳步前进。感谢雷鑫师姐、周晗师姐、赵搏欣师姐、王勋、赵树龙、唐邓清、马兆伟、赵述龙、赵云云、王树源、曹正江,与你们一起在实验室学习生活,在外场加班熬夜的日子终生难忘。感谢周波、王小霞、向邵华、胡豆、李腾翔、习叶勋、吕飞和冯甜甜,我论文试验工作的顺利开展与你们密不可分。无论是在月亮岛的帐篷守夜,还是一起在湘江边用脚手架搭建起浮动平台;无论是软硬件的综合调试,还是每一个架次的安全起飞和降落,所有试验的背后都有你们的辛苦付出与默默支持。

感谢在德国汉堡留学期间,张建伟老师对我的帮助和指导。相对宽松和开放的科研环境,让我不仅幸运的投中两篇IROS和一篇ICRA会议论文,更有机会到瑞典、丹麦、西班牙、日本和香港等科研单位参观见学,能够与一线优秀学者和博士生共同工作、学习和生活。同时也感谢德国实验室肖军浩、孙波和程刚师兄以及管昊君、何俊虎对我的帮助和关心。异国他乡的求学生活,虽然少了很多国内细碎事物的干扰,但增加的是对自我管理的巨大考验。几位师兄不仅告诉我调整学习、生活和工作的方法,更指导我如何融入德国人的生活,关注他们注重效率和质量的生活与工作方式,从而让自己弥补缺点和改善不足。此外,还要感谢侯艺华老师和很多不知姓名的留学基金委老师在公派出国期间给予的各类帮助。出国所提升的不仅仅是学术能力,更是对整个人综合素质的全方位塑造。

感谢我的父母对我的培养和教育。从人类进化和行为认知的角度来看,父母对一个人的影响最大,也是最为深远的。您们常常表扬说我长大了,成熟了,这本质上是您们的榜样作用对我产生的巨大影响,这也是我继续前进的强大动力。快速向前的路上,通常都是逆风,抵御风险的能力也绝不会来自从小到大的庇护。能够面对小概率事件的泰然接纳,面对同一事件正反两个方面的辩证思考,面对不确定风险的镇定自若,都是过往的积累,更是奋进向前的宝贵财富。

身处优美校园,有明确研究课题,有导师和团队倾力相助的研究生活已经结束,然而对于未来世界的探索才刚刚开始。选择怎样的人生,选择怎样的心态,往往并不像推导一个公式,证明一个定理那样,一次成型便清晰明了。当今是一个给予个人充分自由时代,只有不停下学习的脚步,在众多的选择中选择难的那一项,用多维的视角观察和践行,才能真正的实现持续地成长,用实际行动来感谢每一位帮助过我的人,也激励更多的人。

\end{ack}
