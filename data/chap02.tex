\chapter{无人机着舰环境数学建模}
\label{chap:main}
无人机着舰的过程设计到无人机系统和舰船系统两大部分,因此为了研究这两部分直接的关系,必须对降落过程进行适当的数学模型,以利于对无人机控制和引导系统进行数学仿真和物理实验。

\section{无人机系统坐标系定义}

\subsection{系统惯性坐标系}
系统惯性坐标系(Inertial Frame,$\mathcal{F}^i$),该坐标系一般定义位于地球表面的一点,三个轴的方向$(\mathbf{i}^i, \mathbf{j}^i,\mathbf{k}^i)$与地球的北向、动向和指向地心方向相同,即NED坐标系。通常该坐标系定义为飞机的起飞点或降落点,本文中我们选用无人机被舰船引导系统捕获时,舰船的此刻所在的点为原点。

\subsection{无人机惯性坐标系}
无人机惯性坐标系($\mathcal{F}^v$,Vehicle Inertial Frame),该坐标系的原点位于飞机的重心,三轴方向与系统坐标系平行。

无人机第一惯性辅助坐标系($\mathcal{F}^{v1}$ ),该坐标系绕无人机惯性坐标系$\mathbf{k}^v$轴按右手规则旋转得到,其中旋转角度定义为$\psi$,即偏航角(Yaw Angle)。

无人机第二惯性辅助坐标系($\mathcal{F}^{v2}$),该坐标系过绕人机第一惯性辅助坐标系$\mathbf{j}^{v1}$轴按右手规则旋转旋转得到,其旋转角度定义为$\theta$,即俯仰角(Pitch Angle)。

无人机机体坐标系($\mathcal{F}^b$,Body Frame ),该坐标系绕无人机第二惯性辅助坐标系$\mathbf{i}^{v2}$轴按右手规则旋转旋转得到,其旋转角度定义为$\phi$,即横滚角(Roll Angle),有时该角度也被称为倾斜角(Bank Angle)。

无人机稳定坐标系($\mathcal{F}^s$,Stability Frame),该坐标系绕无人机机体坐标系$\mathbf{j}^b$按左手规则旋转得到。其中,定义无人机相对于机体周边空气的速度向量为$\mathbf{V}_a$,其大小为$V_a$。为使机翼产生升力,机翼与风速的夹角必须为正,该角度定义为攻角。这里使用左手系的原因是为更方便的定义定义攻角$\alpha$的正负,即沿稳定坐标系$\mathbf{j^s}$按右手系转动到机体坐标系的角度为正。稳定坐标系的$\mathbf{i}^s$轴与空速向量$\mathbf{V}_a$在$\mathbf{i}^b$-$\mathbf{k}^b$的投影方向平行。

无人机风向坐标系($\mathcal{F}^w$,Wind Frame),该坐标系的$\mathbf{i}^w$轴与风速方向相同,可以通过旋转稳定坐标系的$\mathbf{k}^s$轴$\beta$角度得到,该角度$\beta$被定义为侧滑角。



\section{舰船系统坐标系定义}


\section{舰载引导系统坐标系定义}
 
