\chapter{基于多传感器的无人机自主着舰系统}

\subsection{引导设备选型}
了实现全天候,高精度,复杂应用情况下的通用无人机自主起降引导系统,需要利用多种传感器,并对传感器信息进行融合,为无人机的自主起降提供可靠保障。

参考国内外同类系统设计,考虑通用性、维护性等方面的需求,本系统设计采用了多传感器融合,综合了远程搜索、中近距离精确导航、近距离精确定位的功能,保证能在多种气象条件下,系统安全可靠工作,为无人机的自主起降过程提供丰富精确的导航信息。按照不同的精度要求和成本控制,本系统分为三种不同的配置,以降低对单类型传感器性能指标的要求,提高系统的可靠性。具有快速展开、模块化、方便集成以及高效的多传感器融合的特点,具体需要实现以下功能指标:
(1)能手动和自动切换控制云台,方便系统的标定和调试,可以实现云台快速和精准的指向,以跟踪目标。
(2)能够适应雨雪阴天等各种复杂天气情况,能在可见光和近红外相机之间进行快速切换,方便调试和观察。
(3)当无人机距离地面引导系统大于2km,进入环线飞行距离达到4km时,引导系统通过远红外相机或者UWB定位系统,能够方便的识别、跟踪以及对无人机目标姿态的监视。
(4)地基引导系统能够实时得到并显示转台当前的位置和DGPS坐标值,能够使用户更加直观地判断系统的运行状态,对视觉引导的数据能够进行精度对比分析,当运行不稳定时能及时地通知用户加以修正。
(5)能够完成无人机起飞和降落过程的引导。当无人机需要起飞时,云台自动等待无人机起飞;当无人机降落时,云台转动到降落引导窗口,等待无人机的降落。引导系统开始工作后,系统自动把引导系统得到的位置信息反馈给无人机,引导无人机起降。
(6)引导系统的相对定位精度指标必须达到:水平方向≤0.5m,高度方向≤0.2m。
(7)最大探测距离不低于4km(翼展5米无人机),最大探测距离不低于2km(翼展3米无人机)。
(8)具备昼夜起降引导能力。


\subsection{转台选型}
转台选用美国FLIR公司的双轴陀螺稳定转台,型号为PTU-D300,支持通过RS232串口发送指令,并能实时反馈转台状态,包括转台的偏转角(Pan)和俯仰角(Tilt)。转台的分辨率为 。两侧可以安装负载支架,支持多种安装模式,可以为多种传感器提供搭载平台。PTU-D300为步进电机驱动,能够满足闭环跟踪目标的需求。转台抗振动冲击较好,防护等级IP67,支持野外、舰船、车载等恶劣工作环境。具体参数可参考表 2表 2 PTU-D300转台参数。

无人机的飞行速度为30m/s,当无人机距离引导系统200m时,转台最大转过的速度为4.3°/s。满足转台的速度范围。无人机降落过程中的水平方向转过的角度大概为10m,垂直方向转过的角度大概为50m,从开始降落到降落完成大概15s,以500为计算,转动过程中的角度速度大概为0.076°,也满足最低的转速要求。转台的精度为0.006425°,所以当无人机目标距离引导系统500m时,水平方向的误差为0.0561m,满足系统精度的误差。同时,转台上摆放的起降大概10kg,其中远红外8kg,其它的2kg,也能满足转台的性能要求。
为防止在转台转动过程中底座发生晃动,在二维主动视觉系统的转台的底部设计了可调式三角稳定底座,机械结构如图 5所示,可以增加转台在转动过程中的稳定性。具有较强的负载能力和较好的调节精度,可以满足引导系统的工作要求。

\subsection{可见光相机}
像机选择GigE接口的The Imaging Source彩色CCD高速工业像机DFK 23G445,坚固耐用,适合于野外环境。选用CS接口的100mm固定焦距的镜头,实物如图 6。该相机像素分辨率为640×480,像元尺寸为5.6 m。采集的图像通过千兆网口传送给图像处理计算机,传输距离较远,方便软件开发。摄像机安装100mm焦距镜头后的视场角是 ,深度方向距离为400m时,探测窗口大小约为13.5×10.1m;深度方向距离为800m时,探测窗口大小约为27×20.2m。对于3米翼展的无人机,在800米处的成像像素为71个,能够达到目标识别的需求。因此,为增加探测范围,利用转动平台与摄像机构成主动视觉系统来对无人机目标实时检测-跟踪。

\subsection{红外相机传感器}
红外热像仪是利用红外探测器和光学成像物镜接受被测目标的红外辐射能量,将分布图形反映到红外探测器的光敏元件上,从而获得红外热像图,这种热像图与物体表面的热场分布相对应。通俗地讲红外热像仪就是将物体发出的不可见红外能量转变为可见的热图像,如图 7所示,主要的参数指标如下。
本系统选用的浙江大力科技股份有限公司生产的红外热像仪CM6240FC,大气能见度大于10km,相对湿度小于$80\%$,目标与背景温差3K条件下,对3m×3m目标,探测距离不小于7km,识别距离不小于5km。对1.7m×0.5m目标(人),探测距离不小于5km,识别距离不小于2.5km。
 	最宽视场(40mm焦距):13.75°× 11°;
 	窄视场(240mm焦距):2.3°× 1.8°;
 	连续变焦,变焦过程中保持图像清晰;
 	视场公差:$±5\%$
 	
\subsection{电台设备}	
采用的是加拿大Microhard公司的Nano IP系列的IPn 920(图 8),具有尺寸小、低功耗、传输可靠、能远距离快速通信等优势。其工作频率为902-928MHz,具有 frequency hopping spread spectrum (FHSS) 、 1.2Mbps 的传输速率,digital transmission service (DTS)技术等。能够实现自动组网,有点对点(P2P)、点对多点(P2E)、任意点对任意点(E2E)多种组网模式,同时支持网络、2路串口通信,所以非常适合于无人机控制和命令链路的数据传输。


\subsection{激光超射模块}
使用波长为940nm的半导体激光器,功率为25w,发出连续照明用激光,具体参数见错误!未找到引用源。。在无人机平台前端安装全反射棱镜,使用无觇牌对中杆棱镜ADH11-S,它是仿Topcon角锥棱镜,正面加常数-30mm,背面加常数0mm。全反射棱镜保证了各个方向的光都会按照原方向反射回去,即从转台发射的940nm激光会按照原来方向回到原点。同时,在可见光相机前安装940nm窄带滤光片,使可见光相机只能收到940nm波长的光。

本系统选用山东神戎电子股份有限公司生产的SHR-JLI1000激光照明器,该设备主要分为940mm激光的产生和940mm激光的发射两个部分,如图 9所示,(a)为激光器的产生部件,(b)为激光器的发射部件,激光器的发射部件可以调节激光出射的照明角度,通过调节这个角度可以对目标实现远距离和近距离的有效探测识别,激光器的发射部件同样可以和激光器的产生部件分开,可以方便的安装在需要的位置上。

\subsection{超宽带测距设备}
超宽带无线测距以其高距离的分辨力、强穿透力、低截获率以及很强的抗干扰能力在军事、商业等领域得到越来越多的关注。使用的UWB型号为Time Domain公司的PulsONP410,是P400s的升级版本,可通过加载定向天线增大探测距离(http://www.timedomain.com)。实物如图 11所示。

P410的RF传输频率3.1 GHz 到5.3 GHz,中心频率位于4.3 GHz附近;具有两路用户可调天线端口;单个板子尺寸为7.6 x 8.0 x 1.6 cm;可进入睡眠模式,降低功耗;脉冲发射功率可调,可达10Hz以下。该设备探测距离2km以上,如果选用增强型的定向天线探测距离可以达到5km以上,增强型的定向天线如图12所示。加入定向天线后,UWB的探测波半角为20°,当远距离无人机进入探测窗口时,UWB可以探测到目标。UWB的精度为0.5m,当无人机距离目标500m、水平位置为10m,高度为30m时。水平方向的误差为0.01m,高度为0.03m,满足系统的精度要求。

UWB采用应答的方式接收数据,如图13所示,当地面引导系统把地面的请求发给无人机上的UWB时,无人机上的UWB响应地面请求发送数据,地面上的UWB接收这个信息转化为距离信息。该方式有效的解决了一对一的模式,有利于目标远距离的探测,保证了数据的稳定可靠性。传输频率3.1 GHz 到5.3 GHz,因此UWB不易被干扰。