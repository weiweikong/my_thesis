\chapter{结束语}

\section{论文工作总结}
无人机自主着舰问题是无人机在海面舰艇应用的关键问题之一。本文主要针对上述研究背景展开工作,主要完成工作如下:

(1)提出了一种地基/舰基通用的多传感器无人机回收引导系统。本文通过设计三种不同的地基/舰基引导方案的理论研究和实验测试,提出了由两个独立引导单元组成,基线可调的多传感器无人机降落引导系统。其中,每个引导单元配备一个二自由度转台和可见光相机、红外相机等传感器。两个独立引导单元的排布可以根据目标无人机的大小和检测距离进行优化配置。本文针对该系统独立分布在跑道两侧的特点,通过对目标位置解算理论推导、误差分析和实验验证,证明了在无人机降落过程中,使用上述两种引导系统的可行性。

(2)设计并实现无人机降落过程中实时目标跟踪和位置解算算法。针对引导无人机降落过程中,无人机目标尺度快速变化和姿态未知的问题,本文尝试改进并测试Meanshift、Active Contour、AdaBoost、TLD和CMT等一系列图像跟踪方法,最终设计一套适用于无人机降落过程中的目标跟踪和位置解算方案:通过改进基于形态学滤波的图像预处理方法,TLD目标跟踪框架和基于主动轮廓的目标位置修正方法,结合转台运动位置和无人机运动的估计,能够准确解算出无人机在降落过程中相对于舰船的位置信息,满足无人机引导和控制系统的需要。

(3)设计并实现基于非线性模型预测控制(NMPC)和总能量控制(TESC)的无人机着舰控制系统。由于无人机机载设备运算能力的约束,本文设计了内环控制器和外环控制器来实现无人机的自主降落。其中内环控制器主要由PI和PID控制器组成,主要完成对无人机姿态的控制;外环控制器主要由非线性模型预测控制器(NMPC)和总能量控制器(TESC)组成,针对基于Dubins Path生成的降落曲线进行跟踪。上述算法在仿真环境和Pixhawk和Odroid XU4嵌入系统中均进行了实时性测试。

(4)设计并实现无人机舰载着舰系统仿真环境并进行户外实验验证。本文基于机器人操作系统(Robot Operation System,ROS)和Gazebo仿真环境构建了无人机舰载着陆软件在回路仿真系统(SITL)和硬件在回路仿真系统(HIL)。该仿真环境能够满足上述算法的验证需求。通过二自由度转台与多传感器的组合配置,实现在地面机场和水面环境对小型和中型固定翼无人机的引导和自主降落。

\section{未来工作展望}
无人机舰载降落是一个非常复杂的过程。本文针对该问题提供了一种基于多传感器的舰基引导系统解决方案。在后续的研究中还需要对一下几个方面进行进一步研究和探索:

(1)无人机目标的姿态跟踪和估计。目前对于无人机目标,只是实现了对其简单识别和稳定跟踪。系统的输出结果是对无人机中心点的估计和无人机所在区域的矩形框,对无人机当前的姿态没有明确的描述。由于无人机在降落过程中,如果能够对无人机控制器提供舰载视角的姿态信息,则可以进一步提高对无人机降落过程中引导和控制精度。

(2)无人机与舰载引导系统之间的多传感器融合设计。本文提出的多传感器方案主要是增强地面引导系统对无人机距离检测和目标识别两个方面的性能,没有考虑无人机机载传感器提供的相关信息。当前,无人机机载系统的惯导传感器和图像传感器已经广泛使用,如何有效将舰载引导系统的传感器信息与无人机机载传感器信息有效结合需要进一步研究。

(3)无人机引导控制策略相对简单。本文目前使用的无人机控制策略是基于底层PID控制和顶层NMPC和TECS的控制方法来实现。使用上述方法的主要原因是受限于机载嵌入式系统的运算能力。由于无人机在舰载回收过程中的环境复杂多变,因此目前通用性较强的控制策略并不完全适用,需要进一步研究满足复杂环境且实时性强的控制策略。