\begin{resume}

\section*{作者攻读博士学位期间发表的学术论文} % 发表的和录用的合在一起

[1] \textbf{Weiwei Kong}, Tianjiang Hu, Daibing Zhang, Lincheng Shen, and Jianwei Zhang. Localization Framework for Real-Time UAV Autonomous Landing: An On-Ground Deployed Visual Approach. Sensors 17, no. 6 (2017): 1437. (SCI检索,DOI:10.3390/\\s17061437)

[2] \textbf{Weiwei Kong}, Daibing Zhang, Xun Wang, Zhiwen Xian, and Jianwei Zhang. Autonomous landing of an UAV with a ground-based actuated infrared stereo vision system. In Intelligent Robots and Systems (IROS), 2013 IEEE/RSJ International Conference on, pp. 2963-2970. IEEE, 2013.(EI检索号:20140817338415)

[3] \textbf{Weiwei Kong}, Dianle Zhou, Yu Zhang, Daibing Zhang, Xun Wang, Boxin Zhao, Chengping Yan, Lincheng Shen, and Jianwei Zhang. A ground-based optical system for autonomous landing of a fixed wing UAV. In Intelligent Robots and Systems (IROS), 2014 IEEE/RSJ International Conference on, pp. 4797-4804. IEEE, 2014.(EI检索号:20144800250933)

[4] \textbf{Weiwei Kong}, Dianle Zhou, Daibing Zhang, and Jianwei Zhang. "Vision-based autonomous landing system for unmanned aerial vehicle: A survey." In Multisensor Fusion and Information Integration for Intelligent Systems (MFI), 2014 International Conference on, pp. 1-8. IEEE, 2014.(EI检索号:20150400443875)

[5] \textbf{Weiwei Kong}, Daibing Zhang, and Jianwei Zhang. A ground-based multi-sensor system for autonomous landing of a fixed wing UAV. In Robotics and Biomimetics (ROBIO), 2015 IEEE International Conference on, pp. 1303-1310. IEEE, 2015.(EI检索号:20161802327536)

[6] \textbf{Weiwei Kong}, Zhang Daibing, Zhao Shulong, Zhou Dianle, Zhao Boxin, Zhong Zhiwei, Ma Zhaowei, Tang Dengqing, and Zhang Jianwei. Autonomous track and land a MAV using a modified tracking-learning-detection framework. In Control Conference (CCC), 2015 34th Chinese, pp. 5359-5366. IEEE, 2015. (EI检索号:20154601538704)

[7] \textbf{孔维玮}, 先治文, 张代兵. 无人机垂直起降视觉自主导航系统的设计与实现[A]. 第九届全国信息获取与处理学术会议论文集 [C]. 2011年

[8] Bo Sun, \textbf{Weiwei Kong}, Junhao Xiao, and Jianwei Zhang. A hough transform based scan registration strategy for Mobile Robotic Mapping. In Robotics and Automation (ICRA), 2014 IEEE International Conference on, pp. 4612-4619. IEEE, 2014. 

[9] Tianjiang Hu, Boxin Zhao, Dengqing Tang, Daibing Zhang, \textbf{Weiwei Kong}, and Lincheng Shen. ROS-based ground stereo vision detection: implementation and experiments. Robotics and biomimetics 3, no. 1 (2016): 14.

[10] Yunyun Zhao, Xiangke Wang, \textbf{Weiwei Kong}, Lincheng Shen, and Shengde Jia. Decision-making of UAV for tracking moving target via information geometry. In Control Conference (CCC), 2016 35th Chinese, pp. 5611-5617. IEEE, 2016.

[11] Dengqing Tang, Tianjiang Hu, Lincheng Shen, Daibing Zhang, \textbf{Weiwei Kong}, and Kin Huat Low. Ground stereo vision-based navigation for autonomous take-off and landing of uavs: a chan-vese model approach. International Journal of Advanced Robotic Systems 13, no. 2 (2016): 67.

[12] Shulong Zhao, Xiangke Wang, \textbf{Weiwei Kong}, Daibing Zhang, and Lincheng Shen. A novel data-driven control for fixed-wing UAV path following. In Information and Automation, 2015 IEEE International Conference on, pp. 3051-3056. IEEE, 2015.

[13] Boxin Zhao, Tianjiang Hu, Yifeng Niu, Dengqing Tang, Zhaowei Ma, \textbf{Weiwei Kong}, and Lincheng Shen. Exploring the most appropriate feature detector and descriptor algorithm for on-board UAV image processing. In Information and Automation, 2015 IEEE International Conference on, pp. 56-61. IEEE, 2015.

[14] Hongliang Li, Zhiwei Zhong, \textbf{Weiwei Kong}, and Daibing Zhang. A fast calibration method for autonomous landing of UAV with ground-based multisensory fusion system. In Information and Automation, 2015 IEEE International Conference on, pp. 3068-3072. IEEE, 2015.

[15] 王勋, \textbf{孔维玮}, 张代兵, 朱华勇. 无人机跟踪地面非合作目标的分段引导与控制方法 [J]. 中国科学技术大学学报. 2012 (09)

\section*{作者攻读博士学位期间参加的课题研究} % 有就写,没有就删除

[1] 无人机XXXX着舰技术,装备预先研究项目,核心成员,2011年1月至2015年12月

[2] 无人机自主起降引导控制技术,军民融合协同创新研究项目,核心成员,2013年1月至2016年12月

[3] 机器人操作系统,学校重大应用基础研究项目,核心成员,2015年1月至2017年12月


%\section*{作者攻读博士学位期间撰写的项目报告} 
%[1] 无人机XXXX着舰技术技术报告,装备预先研究项目,2015年12月
%
%[2] 地基引导系统设计报告,军民融合协同创新项目,2016年2月
%
%[3] 控制系统设计报告,军民融合协同创新项目,2016年2月


\end{resume}
