\begin{cabstract}
近年来,随着无人作战系统(Unmanned Combat System,UCS)在伊拉克、阿富汗等几场局部战争的广泛使用,展现了无人作战系统的巨大军事价值。作为无人作战系统中的重要组成,无人机所承担的侦察、监视、通信中继、战场评估、攻击引导等任务也日益增多。为扩大无人机的工作半径,提升作战效能,无人机在舰船上回收的需求更加迫切。虽然无人机在地面起飞、降落或回收已经成为常态,但上述过程中的引导方式主要依赖卫星定位导航系统(Global Navigation Satellite System,GNSS)。在未来“反介入/区域拒止(A2/AD)”的战场环境下,仅仅依赖卫星导航系统无法满足无人机在舰船的降落需求。本文主要针对上述研究背景展开工作,主要完成工作和创新点如下:

(1)提出了一种地基/舰基通用的多传感器无人机回收引导系统。该系统主要有两个独立引导单元组成,每个引导单元配备一个二自由度转台和可见光相机、红外相机等传感器。两个独立引导单元的排布可以根据目标无人机的大小和检测距离进行优化配置。本文针对该系统独立分布在跑道两侧的特点,通过对目标位置解算理论推导、误差分析和实验验证,证明了在无人机降落过程中,使用上述两种引导系统的可行性。

(2)设计并实现无人机降落过程中实时目标跟踪和位置解算算法。针对引导无人机降落过程中,无人机目标尺度快速变化和姿态未知的问题,通过改进基于形态学滤波的图像预处理方法,TLD目标跟踪框架和基于主动轮廓的目标位置修正方法,结合转台运动位置和无人机运动的估计,能够准确解算出无人机在降落过程中相对于舰船的位置信息,满足无人机引导和控制系统的需要。

(3)设计并实现基于非线性模型预测控制(NMPC)和总能量控制(TESC)的无人机着舰控制系统。由于无人机机载设备运算能力的约束,本文设计了内环控制器和外环控制器来实现无人机的自主降落。其中内环控制器主要由PI和PID控制器组成,主要完成对无人机姿态的控制;外环控制器主要由非线性模型预测控制器(NMPC)和总能量控制器(TESC)组成,针对基于Dubins Path生成的降落曲线进行跟踪。

(4)设计并实现针对无人机舰载着陆问题的仿真系统和实验验证系统。本文基于机器人操作系统(Robot Operation System,ROS)和Gazebo仿真环境构建了无人机舰载着陆软件在回路仿真系统(SITL)和硬件在回路仿真系统(HIL)。该仿真环境能够满足上述算法的验证需求。通过选择二自由度转台和响应的传感器,在地面机场和水面实现无人机的自主引导和降落。

上述理论和算法成果分别发表在2012年和2013年的IEEE/RSJ智能机器人与系统国际会议(IROS)会议,得到了领域内同行的认可。基于上述算法的实际飞行测试分别在江西吉安机场和湖南长沙湘江水域进行,验证了基于多传感器的地基引导系统和舰基引导系统的可行性和可靠性。

\end{cabstract}
\ckeywords{无人机;自主着舰;多传感器引导与控制;}

\begin{eabstract}
 In recent years, with the unmanned combat system(UCS) in Iraq, Afghanistan and several other local wars of widespread use, unmanned combat system is showing the great military value. As an important component of unmanned combat system, unmanned aerial vehicles(UAVs) are increasingly assigned with reconnaissance, surveillance, communication relay, battlefield evaluation and attack guidance. In order to expand the operating radius of UAV, improve combat effectiveness, the demand of autonomous landing on a destroyer or carrier is continuing to grow. Although UAVs take off and landing on the standard airport is quite common, navigation in the above maneuvers relies mainly on the Global Navigation Satellite System (GNSS). In the future A2 / AD battlefield environment, the satellite navigation system alone can not meet the UAVs' landing demand. This paper aims at the above research background, the main work and innovation are as follows:
 
(1) A ground / ship-based multi-sensor unmanned aerial vehicle (UAV) recovery and guidance system is proposed. The system consists of two independent guidance units, each with a two degrees of freedom PTU(Pan/Tilt Unit) and visible light camera, infrared camera or other sensors. The configuration of the two independent guidance units can be optimally setup according to the size of the UAV and the proposed detection distance. In this thesis, the feasibility of using the above two guidance systems in the process of UAV landing is proved by the theory of target position solution, error analysis and experimental verification, which is based on the characteristics of the system distributed on both sides of the runway independently.

(2) Design and implement the real-time target tracking and location algorithm in UAV landing process. Aiming at the problem of traking the UAV, size rapidly changing from tiny to large scale, the method of image preprocessing based on morphological filtering, TLD target tracking framework and target position updating based on active contour are improved. During the landing process, the Motion position and unmanned aerial vehicle motion estimation, can accurately calculate the position of UAV relative to the ship's position 

(3) Design and implementation of the non-linear model predictive control (NMPC) and total energy control (TESC) of the UAV landing control system. Due to the computation constraints of UAV on-board equipment, the inner-loop controller and the outer-loop controller are designed to realize the autonomous landing. The inner loop controller is mainly composed of PI controller and PID controller. The outer loop controller mainly consists of nonlinear model predictive controller (NMPC) and total energy controller (TESC) in order to tracking the landing curve which generated by Dubins Path algorithms.

(4) Design and implement the simulation system and experimental verification system for UAV shipboard landing problem. Based on the Robot Operation System (ROS) and Gazebo simulation environment, a software in the loop simulation system (SITL) and hardware in the loop simulation system (HIL) is constructed. The simulation environment can meet the verification requirements of the above algorithms. By selecting PTUs and reltaed sensors, autonomous guidance and landing of the UAV can be achieved at ground and airfields.

The above theoretical and algorithmic results were published in the 2012 IEEE / RSJ Intelligent Robot and Systems International Conference (IROS) meeting in 2012 and were recognized by peers in the field. The actual flight tests based on the above algorithms were carried out in Ji'an Airport of Jiangxi Province and Xiangjiang River of Changsha, Hunan Province, respectively. The feasibility and reliability of the foundation guidance system and ship-based guidance system based on multi-sensor were verified.
 
 
\end{eabstract}
\ekeywords{UAV; Autonomous Landing; Multi-sensor Guidance and Control}

